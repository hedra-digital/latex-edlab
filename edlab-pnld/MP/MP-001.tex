\part{Manual do professor}

\begin{flushright}
\textsc{ieda lebensztayn}
\end{flushright}

\chapter{Carta ao Professor}

Prezada professora, prezado professor,

Com muita alegria, apresentamos a você a antologia \emph{As calças do
Raposo e outros causos de ilusões e confusões}. Cada um dos contos aqui
reunidos tem um interesse específico e, juntos, eles compõem um conjunto
bastante fértil para análise em sala de aula. Alguns deles contêm
suspense, outros despertam o riso, alguns causam comoção, e outros
pressupõem conhecimentos de áreas como Biologia e História, incitando em
especial o estudo interdisciplinar. Todos levam a pensar sobre as
ilusões e desilusões que carregamos na vida, falam de sonhos e
possibilidades de realizá-los, de enganos, mentiras e confusões
frequentes no comportamento humano, nas relações sociais. Daí seu
sentido ético e político: deixam ver a necessidade de respeito entre as
pessoas, de se considerarem as diferenças, as singularidades, contra
preconceitos. Trata-se de clássicos: textos para serem lidos em classe e
em todo canto, afinal contribuem para a compreensão das pessoas e da
realidade --- o que é fundamental hoje e sempre.

Os contos estão dispostos em cinco seções, conforme assuntos do
cotidiano, do universo escolar e de dimensões afetivas, políticas e
culturais, que certamente vão despertar seu interesse: ``Na Escola'',
``Na Realidade Social'', ``Nas Relações Amorosas'', ``Na Língua'' e
``Entre Animais''. A autoria é de grandes escritores da literatura
brasileira: Medeiros e Albuquerque, Machado de Assis, Artur Azevedo,
Alberto de Oliveira, Monteiro Lobato e Lima Barreto.

Por esses e outros motivos que apresentaremos a seguir, a leitura, análise e interpretação de \emph{As calças do
Raposo e outros causos de ilusões e confusões} contempla a décima primeira Competência Específica de Língua Portuguesa da Base Nacional Comum Curricular (BNCC): 

"Envolver-se em práticas de leitura literária que possibilitem o desenvolvimento do senso estético para fruição, valorizando a literatura e outras manifestações artístico-culturais como formas de acesso às dimensões lúdicas, de imaginário e encantamento, reconhecendo o potencial transformador e humanizador da experiência com a literatura". 

Também contemplamos, evidentemente, a seguinte Habilidade:

(EF89LP33) Ler, de forma autônoma, e compreender – selecionando procedimentos e estratégias de leitura adequados a diferentes objetivos e levando em conta características dos gêneros e suportes – romances, contos contemporâneos, minicontos, fábulas contemporâneas, romances juvenis, biografias romanceadas, novelas, crônicas visuais, narrativas de ficção científica, narrativas de suspense, poemas de forma livre e fixa (como haicai), poema concreto, ciberpoema, dentre outros, expressando avaliação sobre o texto lido e estabelecendo preferências por gêneros, temas, autores.

Dessa forma, neste material de apoio, você encontrará:

\begin{itemize}
\item Uma apresentação dos autores e do contexto de produção dos contos, no
que tange a aspectos sociais, culturais, temporais e geográficos, para
que você possa aprofundar-se na antologia;

\item Uma apresentação do contexto de recepção dos textos da antologia, para você criar estratégias de uso da obra em sua aula a partir de impressões
de leitores, especializados e não especializados;

\item Descrição do gênero a que pertencem os textos da antologia;

\item Propostas de Atividades, formuladas nos termos da Base
Nacional Comum Curricular, divididas em pré-leitura, leitura e pós-leitura;

\item Sugestões de referências complementares, com diversas fontes de
análise, e breve bibliografia comentada, com referências 
para que você possa preparar sua aula com profundidade.
\end{itemize}

Bom trabalho!

\chapter{Apresentação dos autores e dos contextos de produção e de recepção da obra}

\section{A organizadora}

\textbf{Ieda Lebensztayn} é crítica literária, pesquisadora e ensaísta. Mestre em Teoria Literária e doutora em Literatura Brasileira pela \textsc{usp}. Fez pós-doutorado no Instituto de Estudos Brasileiros (\textsc{ieb-usp}) e na Biblioteca Brasiliana Mindlin / Faculdade de Filosofia, Letras e Ciências Humanas (\textsc{bbm/fflch-usp}). Autora de \emph{Graciliano Ramos e a Novidade: o astrônomo do inferno e os meninos impossíveis} (Hedra, 2010). Organizou, com Hélio de Seixas Guimarães, os dois volumes de \emph{Escritor por escritor: Machado de Assis segundo seus pares} (Imesp, 2019). E, com Thiago Mio Salla, os livros \emph{Cangaços}, \emph{Conversas} (Record, 2014) e \emph{O antimodernista: Graciliano Ramos e 1922} (Record, 2022).

\section{A ilustradora}

\textbf{Ana Lancman} nasceu em 1995, em São Paulo. Formou-se em Ciências Sociais na USP e trabalha com edição de livros, design gráfico e ilustração. Publicou de forma independente os livros \emph{Qual pedra abraçar para não ser levada pela correnteza} e \emph{Costuro plutão}.

\section{Autores}

José Joaquim de Campos da Costa de \textbf{Medeiros e Albuquerque}
(Recife, Pernambuco, 1867--Rio de Janeiro, Rio de Janeiro, 1934) foi
romancista, contista, poeta, teatrólogo, ensaísta, memorialista,
jornalista, professor, político e orador. Fundou a Cadeira número 22 da
Academia Brasileira de Letras.

Cursou o Colégio Pedro \textsc{ii} e a Escola Acadêmica, em Lisboa. De volta ao
Rio de Janeiro, fez um curso de História Natural com Emílio Goeldi e foi
aluno particular do crítico Sílvio Romero. Trabalhou como professor
primário adjunto, entrando em contato com escritores como Paula Ney e
Pardal Mallet. Em 1888, colaborou no jornal \emph{Novidades}. Estreou em
1889 com os livros de poesia \emph{Pecados} e \emph{Canções da
decadência}.

Republicano, foi nomeado secretário do Ministério do Interior e
vice-diretor do Ginásio Nacional. Lecionou na Escola de Belas Artes
(desde 1890) e em escolas de segundo grau, de 1890 a 1897, além de ter
sido presidente do Conservatório Dramático (1890--1892). É o autor da
letra do Hino da República, cujo refrão é: ``Liberdade! Liberdade!/ Abre
as asas sobre nós/ Das lutas na tempestade/ Dá que ouçamos tua voz''.
Durante o período florianista, dirigiu o jornal \emph{O Fígaro}. Em
1894, foi eleito deputado federal por Pernambuco, conseguindo a votação
para a lei dos direitos autorais. Nomeado diretor geral da Instrução
Pública do Distrito Federal em 1897, por estar na oposição a Prudente de
Morais, teve de pedir asilo à Embaixada do Chile. Demitido, recorreu aos
tribunais e obteve a reintegração. De 1912 a 1916, viveu em Paris. De
volta ao Brasil, de 1899 a 1917 ocupou a Secretaria Geral da \textsc{abl}. Foi
autor da primeira reforma ortográfica, promovida em 1902. De 1930 a
1934, colaborou na \emph{Gazeta} de São Paulo e em jornais do Rio de
Janeiro e na Comissão do Dicionário e na \emph{Revista da Academia}.

Principais obras: poesia: \emph{Pecados} (1889); \emph{Canções da
decadência} (1889); \emph{Poesias}, 1893--1901 (1904); \emph{Fim} (1922);
\emph{Poemas sem versos} (1924); contos: \emph{Um homem prático} (1898);
\emph{Mãe Tapuia} (1900); \emph{Contos escolhidos} (1907); \emph{O
assassinato do general} (1926); \emph{Se eu fosse Sherlock Holmes}
(1932); romances: \emph{Marta} (1920); \emph{Mistério} (1921);
\emph{Laura} (1933); teatro: \emph{O escândalo}, drama (1910) e
\emph{Teatro meu\ldots{} e dos outros} (1923); ensaios e conferências:
\emph{O silêncio é de ouro} (1912); \emph{Pontos de vista} (1913);
\emph{O hipnotismo} (1921); \emph{Graves e fúteis} (1922);
\emph{Literatura alheia} (1914); \emph{Páginas de crítica} (1920);
\emph{Homens e coisas da Academia} (1934); memórias: \emph{Minha vida:
da infância à mocidade, 1867--1893} (1933); \emph{Minha vida: da mocidade
à velhice, 1893--1934} (1934); \emph{Quando eu era vivo\ldots{}}
Memórias, 1867 a 1934 (1942); polêmicas e política: \emph{Polêmicas}.
Coligidas e anotadas por Paulo de Medeiros e Albuquerque (1941);
\emph{Parlamentarismo e presidencialismo} (1932).

Joaquim Maria \textbf{Machado de Assis} (Rio de Janeiro, Rio de Janeiro,
1839--Rio de Janeiro, Rio de Janeiro, 1908) foi jornalista, romancista,
contista, cronista, poeta e teatrólogo. Fundador da cadeira n. 23 da
Academia Brasileira de Letras, ocupou por mais de dez anos a presidência
da instituição, que passou a ser chamada também de Casa de Machado de
Assis.

Filho do pintor afrodescendente Francisco José de Assis e da açoriana
Maria Leopoldina Machado de Assis, perdeu a mãe muito cedo e foi criado
no morro do Livramento. Autodidata, aos quinze anos incompletos publicou
o soneto ``À Ilma. Sra. D.P.J.A.'', seu primeiro trabalho literário,
no \emph{Periódico dos Pobres}. Aprendiz de tipógrafo, entrou em 1856
para a Imprensa Nacional, onde conheceu o escritor Manuel Antônio de
Almeida, que se tornou seu protetor. Em 1858, era revisor e colaborador
no \emph{Correio Mercantil} e, dois anos depois, a convite de Quintino
Bocaiuva, entrou para a redação do \emph{Diário do Rio de Janeiro}. Na
revista \emph{O Espelho}, estreou como crítico teatral, colaborou também
na \emph{Semana Ilustrada} e no \emph{Jornal das Famílias}, sobretudo
com contos.

A tradução de \emph{Queda que as mulheres têm para os tolos} (1861) foi
o primeiro livro publicado por Machado de Assis. Em 1862, ele atuou como
censor teatral. Colaborou em \emph{O Futuro}, órgão dirigido por
Faustino Xavier de Novais, irmão de Carolina Augusta Xavier de Novais,
sua futura esposa, com quem se casou em 1869.

Em 1864, saiu seu primeiro livro de poesia, \emph{Crisálidas}; e em
1872, o primeiro romance, \emph{Ressurreição}. No ano seguinte, ele foi
nomeado primeiro oficial da Secretaria de Estado do Ministério da
Agricultura, Comércio e Obras Públicas. Era o início da carreira de
burocrata, sua fonte principal de sustento. Em 1874, o romance \emph{A
mão e a luva foi estampado} em \emph{O Globo}, em folhetins.
Intensificando sua colaboração em periódicos como \emph{O
Cruzeiro}, \emph{A Estação}, \emph{Revista Brasileira}, Machado publicou
contos, romances, crônicas e poemas em folhetins e depois em livros. Sua
peça \emph{Tu, só tu, puro amor} foi encenada no Imperial Teatro Dom
Pedro \textsc{ii} em 1880, em comemoração ao tricentenário de Camões. De 1881 a
1897, publicou na \emph{Gazeta de Notícias} suas melhores crônicas. Em
1880, o poeta Pedro Luís Pereira de Sousa o convidou para oficial de
gabinete no Ministério da Agricultura, Comércio e Obras Públicas. Em
1889, Machado foi promovido a diretor da Diretoria do Comércio no
Ministério em que servia.

De 15 de março a 15 de dezembro de 1880, as \emph{Memórias póstumas de
Brás Cubas} saíram em folhetins na \emph{Revista Brasileira} e, no ano
seguinte, em livro. Tal obra inicia a chamada segunda fase da trajetória
literária de Machado de Assis. Escritor dedicado ao longo de 54 anos de
vida literária a praticamente todos os gêneros --- poesia, crítica,
teatro, tradução, crônica, conto e romance ---, ele exerceu em sua poesia
o romantismo em \emph{Crisálidas} (1864) e \emph{Falenas} (1870), o
indianismo em \emph{Americanas} (1875) e o parnasianismo
em \emph{Ocidentais} (1901). Dessa fase considerada romântica são também
as coletâneas \emph{Contos fluminenses} (1870) e \emph{Histórias da
meia-noite} (1873) e os romances \emph{Ressurreição} (1872), \emph{A mão
e a luva} (1874), \emph{Helena} (1876) e \emph{Iaiá Garcia} (1878). A
partir daí, Machado de Assis criou obras-primas, que fogem a qualquer
denominação de escola literária e o tornaram o grande escritor
brasileiro.

Depois de \emph{Memórias póstumas de Brás Cubas} (1881), publicou os
romances \emph{Quincas Borba} (1890), \emph{Dom Casmurro} (1900),
\emph{Esaú e Jacó} (1904) e \emph{Memorial de Aires} (1908).
Diversamente do viés romântico de seus primeiros romances, essa segunda
fase sobressai por seu realismo pleno de ambiguidades e implacável, o
qual, ao analisar a psicologia das personagens em sociedade,
desestabiliza certezas do leitor e deixa ver a primazia da natureza
egoísta. Inclui volumes de contos como \emph{Papéis avulsos} (1882),
\emph{Histórias sem data} (1884), \emph{Várias histórias} (1896) e
\emph{Páginas recolhidas} (1899).

\textbf{Artur} Nabantino Gonçalves de \textbf{Azevedo} (São Luís,
Maranhão, 1855--Rio de Janeiro, Rio de Janeiro, 1908) foi jornalista,
contista, teatrólogo e poeta. Figurou, ao lado do irmão, o romancista
Aluísio Azevedo, no grupo fundador da \textsc{abl} e criou a Cadeira número 29,
que tem como patrono o também dramaturgo Martins Pena. Fundou
publicações literárias, como \emph{A Gazetinha}, \emph{Vida Moderna} e
\emph{O Álbum}. Colaborou em \emph{A Estação}, ao lado de Machado de
Assis, e no jornal \emph{Novidades}, em que seus companheiros eram
Alcindo Guanabara, Moreira Sampaio, Olavo Bilac e Coelho Neto. Escreveu
principalmente sobre teatro em \emph{O País}, no \emph{Diário de
Notícias} e em \emph{A Notícia}. Utilizou diversos pseudônimos: Elói o
Herói, Gavroche, Petrônio, Cosimo, Juvenal, Dorante, Frivolino, Batista
o trocista, e outros. Principais obras de teatro: \emph{Amor por
anexins} (1872); \emph{A pele do lobo} (1877); \emph{A}
\emph{almanjarra} (1888); \emph{A Capital Federal} (1897);
\emph{Confidências} (1898); \emph{O jagunço} (1898). Poesias:
\emph{Sonetos} (1876); Rimas (1909). Contos: \emph{Contos possíveis}
(1889); \emph{Contos fora de moda} (1894); \emph{Contos efêmeros}
(1897); \emph{Contos cariocas} (1928).

Antônio Mariano \textbf{Alberto de Oliveira}
(Saquarema, Rio de Janeiro, 1857--Niterói, Rio de Janeiro,
1937) foi poeta, professor e farmacêutico. Atuou como secretário
estadual de educação, foi membro honorário da Academia de Ciências de
Lisboa e fundador da Academia Brasileira de Letras. Formado em Farmácia
em 1884, cursou até o terceiro ano a Faculdade de Medicina. Havendo
estreado com \emph{Canções românticas}, já nas \emph{Meridionais} (1884)
e nas quatro séries de \emph{Poesias} (1900, 1905, 1913 e 1928) filia-se
ao parnasianismo, bem como Raimundo Correia e Olavo Bilac. Eleito, em
1924, ``príncipe dos poetas brasileiros''. Fundador da Cadeira 18 da
Academia Brasileira de Letras. Publicações: \emph{Canções românticas}
(1878); \emph{Meridionais} (1884); \emph{Relatório do Diretor da
Instrução do Estado do Rio de Janeiro} (1893, 1895); \emph{Versos e
rimas} (1895); \emph{Poesias} (1900, 1905, 1913 e 1928); \emph{O culto
da forma na poesia brasileira} (1916).

José Bento Renato \textbf{Monteiro Lobato} (Taubaté, São Paulo, 1882--São Paulo, São Paulo, 1948) foi pioneiro no Brasil como criador da
literatura infanto-juvenil, do Sítio do Picapau Amarelo, e editor, além
de ter sido contista, jornalista, tradutor, pintor e fotógrafo. Aos onze
anos, mudou seu nome para José Bento, por causa das iniciais gravadas no
castão da bengala do pai, J.B.M.L. Apesar de sua inclinação para as
artes plásticas, cursou a Faculdade de Direito do Largo São Francisco,
em São Paulo, por imposição do avô, o Visconde de Tremembé.

Formado em 1904, voltou a Taubaté, onde foi nomeado promotor público
interino e transferido, em 1907, para Areias, São Paulo. Enviou artigos
para \emph{A Tribuna}, de Santos, traduções para o jornal \emph{O Estado
de S. Paulo} e caricaturas para a revista \emph{Fon-Fon!}, do Rio de
Janeiro. Em 1911, herdou, com as duas irmãs, a fazenda do avô. Publicou,
em 1914, os artigos ``Velha praga'' e ``Urupês'' em \emph{O Estado de S.
Paulo}, criando o personagem Jeca Tatu. Em 1917, vendeu a fazenda e se
mudou para São Paulo. Escreveu, em \emph{O Estado de S. Paulo}, o artigo
``A propósito da Exposição de Malfatti'', que abriu polêmica com os
modernistas.

Em 1918, estreou com o livro de contos \emph{Urupês}, que esgotou 30 mil
exemplares entre 1918 e 1925, e comprou a \emph{Revista do Brasil},
lançando as bases da indústria editorial no país. Criando uma rede de
distribuição, com vendedores autônomos e consignatários, revolucionou o
mercado livreiro. Em 1920, fundou a editora Monteiro Lobato \& Cia. E
lançou \emph{A menina do narizinho arrebitado}, primeira da série de
histórias com que Lobato criou a literatura brasileira dedicada às
crianças, formando gerações de leitores. Em 1924, com capital ampliado e
nova denominação, Companhia Gráfico-Editora Monteiro Lobato, sua editora
montou o maior parque gráfico da América Latina. Porém, no ano seguinte,
dificuldades financeiras o levaram a vender a \emph{Revista do Brasil} e
liquidar a editora. Mudou-se para o Rio de Janeiro e fundou a Companhia
Editora Nacional.

Adido comercial em Nova York de 1927 até 1930, voltou ao Brasil com
ideias para a exploração de ferro e petróleo. Fundou empresas de
prospecção, mas, contrariando interesses multinacionais e fazendo
oposição, em artigos e entrevistas, ao governo Vargas, foi preso por
seis meses em 1941. Recebeu indulto depois de cumprir metade da pena,
mas o governo mandou apreender e queimar seus livros infantis. Em 1944,
recusou indicação para a Academia Brasileira de Letras. Em 1946,
tornou-se sócio da editora Brasiliense. Embarcou para a Argentina e
fundou em Buenos Aires a Editorial Acteon, retornando no ano seguinte a
São Paulo.

Principais obras: livros para crianças: \emph{O Saci} (1921);
\emph{Fábulas} (1922); \emph{Reinações de Narizinho} (1931);
\emph{Viagem ao céu} (1932); \emph{Caçadas de Pedrinho} (1933);
\emph{História do Mundo para as} \emph{Crianças} (1933); \emph{Emília no
País da Gramática} (1934); \emph{Aritmética da Emília} (1935);
\emph{Memórias da Emília} (1936); \emph{O Poço do Visconde} (1937);
\emph{O Picapau Amarelo} (1939); \emph{A Reforma da Natureza} (1941);
\emph{A Chave do Tamanho} (1942); \emph{Os doze trabalhos de Hércules},
dois volumes (1944). Livros para adultos: \emph{Urupês} (1918);
\emph{Cidades} \emph{mortas} (1919); \emph{Ideias de Jeca Tatu} (1919);
\emph{Negrinha} (1920); \emph{Mundo da lua} (1923); \emph{O Presidente
Negro/O choque das raças} (1926); \emph{Ferro} (1931); \emph{América}
(1932); \emph{O escândalo do petróleo} (1936); \emph{A barca de Gleyre}
(1944).

Monteiro Lobato exaltou a ``inteligência criadora'' de Machado de Assis,
que, tendo ascendido socialmente, foi capaz de observar os mecanismos da
sociedade e atingiu a ``intuição perfeita de tudo''. Ele declarou, com
bom humor, que, diante de Machado, ``somos todos uns bobinhos''. Já Lima
Barreto, embora reconhecesse os méritos de grande escritor do criador de
Brás Cubas, tinha ressalvas a sua secura de alma e a sua falta de
simpatia.

Afonso Henriques de \textbf{Lima Barreto} (Rio de Janeiro, Rio de
Janeiro, 1881--Rio de Janeiro, Rio de Janeiro, 1922) foi jornalista e
escritor. Filho de um tipógrafo e de uma professora primária, ambos
descendentes de escravos, ficou órfão de mãe aos sete anos. Proclamada a
República, seu pai foi demitido da Imprensa Nacional, tendo lá entrado
pela mão do Visconde de Ouro Preto. Lima Barreto estudou no Colégio
Pedro \textsc{ii} e ingressou na Escola Politécnica, mas abandonou o curso de
Engenharia, quando seu pai enlouqueceu e foi internado. Para sustentar a
família, passou a trabalhar como amanuense na Secretaria da Guerra em
1903 e colaborava em diversos jornais do Rio de Janeiro. Colaborou no
\emph{Correio da Manhã} e nas revistas \emph{Fon-Fon}, \emph{A.B.C.} e
\emph{Careta}. Editou a \emph{Revista Floreal}, nos fins de 1907.

Iniciou sua carreira de romancista em 1909, com \emph{Recordações do
escrivão Isaías Caminha}. Dois anos depois, \emph{Triste fim de
Policarpo Quaresma} sai em folhetim no \emph{Jornal do Commercio}, e em
1915 em livro. Em 1919, Lima publica \emph{Vida e morte de M. J. Gonzaga
de Sá}, pela editora Revista do Brasil, de Monteiro Lobato. Vivendo
crises de depressão e entregando-se à bebida, internou-se no Hospício
Nacional em 1914 e em 1919, daí ter escrito \emph{Cemitério dos vivos}.
A partir de 1918, passou a militar na imprensa maximalista.

Muitos de seus escritos foram redescobertos e publicados em livro após
sua morte, por Francisco de Assis Barbosa e outros pesquisadores. Outras
publicações: \emph{As aventuras do Dr. Bogoloff} (1912); \emph{Numa e a
ninfa} (1915); \emph{Histórias e sonhos} (1920); \emph{Os Bruzundangas}
(1923); \emph{Clara dos Anjos} (póstumo, 1948); \emph{Diário íntimo}
(1953); \emph{Feiras e Mafuás} (1953); \emph{Cemitério dos vivos}
(póstumo e inacabado, 1956).

\section{Contexto de produção da obra}

Na época de Machado de Assis, Medeiros e Albuquerque, Artur Azevedo e
Alberto de Oliveira, o contexto histórico do país se encaminhou para a
Abolição da Escravatura, em maio de 1888, e a Proclamação da República,
em novembro do ano seguinte. Desde a extinção do tráfico de escravos, em
1850, tinha aumentado a decadência da economia açucareira, deslocou-se o
eixo de prestígio para o Sul, e se intensificaram as ideias liberais,
abolicionistas e republicanas. Mas, como se sabe, a Abolição não
resolveu o problema da espoliação dos ex-escravos, e tal origem colonial
do Brasil trouxe como consequências a violência, a desigualdade social,
a educação precária que marcam o país.

Como se viu, esses quatro escritores fazem parte do grupo
de fundadores da Academia Brasileira de Letras. A fundação da \textsc{abl}, a 28
de janeiro de 1897, contou também com Coelho Neto, Graça Aranha, Joaquim
Nabuco, José do Patrocínio, José Veríssimo, Lúcio de Mendonça, Olavo
Bilac, Rodrigo Otávio, Rui Barbosa, entre outros.

Em termos de sociabilidade literária, a ideia de criar a Academia
Brasileira de Letras veio do grupo de intelectuais que se reunia na
redação da \emph{Revista Brasileira}. Machado de Assis compareceu às
reuniões preparatórias e, quando se fundou a Academia, a 28 de janeiro
de 1897, foi eleito presidente, tendo-se dedicado a ela até o fim da
vida.

Os demais contistas desta antologia, Monteiro Lobato e Lima Barreto,
nascidos respectivamente em 1882 e 1881, justamente o ano de publicação
das \emph{Memórias póstumas de Brás Cubas}, são, portanto, de outra
geração. Se a historiografia enquadra os primeiros na escola realista,
tendo em Machado de Assis seu principal representante, costuma
considerar Lobato e Lima como pré-modernistas. No entanto, tal
denominação, mesmo reconhecendo neles seu teor crítico e revolucionário,
peca por certo anacronismo, ao projetar valor na obra desses autores por
um sentido antecipador dos modernistas de 1922.

Ressalte-se que os escritores aqui reunidos colaboraram bastante na
imprensa, e muitos de seus contos saíram primeiro em periódicos, mais
acessíveis ao público, e depois no formato livro, às vezes sofrendo
modificações. Machado de Assis, por exemplo, publicou nos inícios da
edição e do comércio livreiro no Brasil, com os pioneiros da atividade
no país: Francisco de Paula Brito, os irmãos Garnier, os irmãos Laemmert
e Henri Lombaerts.

Os cinco contos de Machado incluídos nesta antologia foram estampados,
antes da edição em livros, na \emph{Gazeta de Notícias}. ``Galeria
póstuma'' a 2 de agosto de 1883, e ``Primas de Sapucaia!'' a 24 de
outubro de 1883; e em 1884 saíram no volume \emph{Histórias sem data},
editado por B. L. Garnier. A \emph{Gazeta de Notícias} publicou ``Conto
de escola'' a 8 de setembro de 1884, e ``A cartomante'' a 28 de novembro
de 1884, depois reunidos em \emph{Várias histórias} (1896), por Laemmert
\& C. Editores. E o mesmo jornal trouxe ``Ideias de canário'' a 15 de
janeiro de 1895, presente depois em \emph{Páginas recolhidas} (1899), de
H. Garnier, Livreiro-Editor.

Também a \emph{Gazeta de Notícias} deu a público o conto ``Os brincos de
Sara'', de Alberto de Oliveira, a 20 de junho de 1892.

Diário carioca fundado em agosto de 1875 por José Ferreira de Sousa
Araújo, a \emph{Gazeta de Notícias} foi um dos principais jornais da
capital federal durante a chamada República Velha. Introduziu inovações
na imprensa brasileira, como o uso do clichê, das caricaturas e da
técnica de entrevistas. Teve inicialmente por propósito lutar pela
abolição da escravatura e pela proclamação da República. Ferreira de
Araújo formou uma equipe com figuras de destaque na vida pública da
época, como Quintino Bocaiuva e José do Patrocínio. Por volta de 1880,
passou a publicar folhetins, sobretudo traduzidos de autores franceses.
Durante esses primeiros anos, o jornal tinha entre oito e vinte páginas.

Passados dez anos, a \emph{Gazeta de Notícias} se tornou uma sociedade
anônima, e Ferreira de Araújo permaneceu ainda por um tempo na direção.
Pouco depois, foi para a Europa, vendo nos adventos da Abolição e da
República a consolidação de seus principais anseios. Iniciada a
República, o jornal passou a se identificar com a situação, como órgão
antimonarquista e depois como defensor das elites agrárias.

Outro importante periódico, a \emph{Revista Brasileira}, do Rio de
Janeiro, publicou o conto ``As calças do Raposo'', de Medeiros e
Albuquerque, na edição de janeiro-março de 1899 (ano \textsc{v}, tomo \textsc{xvii}, pp.
319--34). Depois, ele foi incluído no volume \emph{Mãe Tapuia}, publicado
no Rio de Janeiro por H. Garnier, Livreiro Editor, em 1900.

Nas páginas da \emph{Revista Brasileira}, como já se apontou, os
leitores conheceram as \emph{Memórias póstumas de Brás Cubas}, de
Machado de Assis. Era a segunda fase do periódico (jun. 1879--dez. 1881),
editada por Nicolau Midosi. Já o conto de Medeiros e Albuquerque saiu na
terceira fase (jan. 1895--set. 1899), dirigida por José Veríssimo. A
Revista funcionava na rua do Ouvidor, 66, endereço em que se reuniam os
fundadores da Academia Brasileira de Letras. E suas páginas apresentaram
também os discursos proferidos na sessão inaugural pelo presidente
Machado de Assis e pelo secretário-geral Joaquim Nabuco, assim como a
``Memória histórica'' do primeiro-secretário Rodrigo Otávio.

Além de livros de Machado e de Medeiros e Albuquerque, a Garnier
publicou também, em 1893, \emph{Contos fora da moda}, de Artur Azevedo,
com ``Útil inda brincando'' e ``Plebiscito''. Os textos igualmente foram
estampados em periódicos: \emph{O Álbum}, do Rio de Janeiro, em abril de
1893, e \emph{O Fluminense}, de Niterói, a 11 e 12 de janeiro de 1899,
trouxeram ``Útil inda brincando''. Curiosamente, essa expressão do
título, bastante comum na época, era legenda de uma escultura no
chafariz dos Jacarés, a Fonte do Menino, no Passeio Público do Rio de
Janeiro. O menino da fonte era lúdico e útil, por fornecer água a todos.

Os jornais \emph{O País}, do Rio de Janeiro, a 4 de abril de 1890,
\emph{O Republicano} de Sergipe, a 3 de maio de 1890, e \emph{O
Republicano} de Cuiabá, a 9 de fevereiro de 1896, publicaram
``Plebiscito''. E \emph{A República}, do Rio de Janeiro, a 13 de julho
de 1894, provocou o riso dos leitores com ``As asneiras do Guedes''.
Observem-se os títulos dos jornais, alusivos ao então novo regime
político. Depois, em 1928, esses textos foram recolhidos no volume
póstumo \emph{Contos cariocas} (Rio de Janeiro: Leite Ribeiro).

Com suas especificidades de estilo, os contistas Lima Barreto e Monteiro
Lobato trazem para a literatura brasileira das primeiras décadas do
século \textsc{xx}, dos inícios da República (sobretudo Lima, que morreu jovem,
em 1922; Lobato também na Segunda República), personagens de baixa
extração social e um olhar atento aos anseios, necessidades e
frustrações dessas figuras, em uma sociedade marcada por desigualdades
sociais.

De Lima Barreto, ``Um bom diretor'' saiu na revista \emph{Careta}, do
Rio de Janeiro, em 3 de abril de 1915, e depois no volume
\emph{Histórias e sonhos. Contos} (São Paulo: Brasiliense, 1961). E ``O
homem que sabia javanês'' foi estampado na \emph{Gazeta da Tarde}, do
Rio de Janeiro, em 28 de abril de 1911, e em 1915, como Apêndice da
primeira edição de \emph{Triste fim de} \emph{Policarpo Quaresma} (Rio
de Janeiro: Tipografia ``Revista dos Tribunais'').

Por fim, de Monteiro Lobato, o texto ``Pé no chão'', de ``Vidinha
ociosa'' (1908), consta de \emph{Cidades mortas}, publicado em 1919,
Edição da Revista do Brasil, de São Paulo. O conto ``Tragédia dum capão
de pintos'' saiu na edição de julho de 1923 da \emph{Revista do Brasil}
e foi recolhido no volume \emph{O macaco que se fez homem}, publicado no
mesmo ano pela editora Monteiro Lobato, de São Paulo.

``Dona Expedita'', de 1939, foi incluído na segunda edição de
\emph{Negrinha}, volume 3 das \emph{Obras completas} de Lobato,
publicadas em 1946 pela Brasiliense, de São Paulo. À primeira edição
desse livro, de 1920, composta dos contos ``Negrinha'', ``As fitas da
vida'', ``O drama da geada'', ``Bugio moqueado'', ``O jardineiro
Timóteo'' e ``O colocador de pronomes'', Lobato acrescentou ``Dona
Expedita'', ``O fisco (Conto de Natal)'' e ``Os pequeninos'', aqui
presentes, e outros contos.

\section{Contexto de recepção da obra}

O conto que dá título a este livro recebeu a melhor das apreciações de
Machado de Assis. Segundo relata seu autor, Medeiros e Albuquerque, ``As
calças do Raposo'' foi lido pelo amigo na \emph{Revista Brasileira}.
Dias depois, quando ia tomar o bonde, no Largo da Carioca, Machado viu
um homem que lhe pareceu conhecido. Fez um grande esforço de memória e
conseguiu identificar quem era: ``--- Ah! É o Raposo, do Medeiros!''. Ou
seja, Machado visualizava com grande intensidade as figuras literárias
que encontrava nos livros, e a caracterização e a história do Raposo
ficaram vivos em seu espírito.

Já Medeiros e Albuquerque, quando do lançamento de \emph{Relíquias de
casa velha}, afirmou, em \emph{A Notícia}, do Rio de Janeiro, em 1906:
``Um livro de Machado de Assis é sempre uma festa para a literatura
nacional''. Entendia que ``a beleza e a sedução'' dos livros do autor de
\emph{Dom} \emph{Casmurro} (para ele, o melhor) reside sobretudo na
graça do estilo: ``leve, irônico, sutil, disfarçando as observações mais
profundas sob frases breves e despretensiosas''.

Criador de obras-primas em especial do conto e do romance, Machado de
Assis soma mais de 160 anos de fortuna crítica acumulada e incessante.
Amigo que conviveu com ele durante 34 anos de trabalho na mesma
repartição pública, Artur Azevedo considerava a obra machadiana um
tesouro a ser guardado e transmitido às gerações futuras. Monteiro
Lobato exaltou a ``inteligência criadora'' de Machado, que, tendo
ascendido socialmente, foi capaz de observar os mecanismos da sociedade
e atingiu a ``intuição perfeita de tudo''. Ele declarou, com bom humor,
que, diante de Machado, ``somos todos uns bobinhos''. Já Lima Barreto,
embora reconhecesse os méritos de grande escritor do criador de Brás
Cubas, tinha ressalvas a sua secura de alma e a sua falta de simpatia.
Essa mostra já reúne motivos recorrentes da fortuna crítica de Machado
de Assis: a admiração pela agudez do observador da realidade e do
analista de caracteres, que ascendeu socialmente e se fez presidente da
\textsc{abl}; as reservas a certo absenteísmo machadiano.

Contemporaneamente, o professor, crítico e poeta Alcides Villaça, ao
analisar ``A cartomante'', apreendeu Machado como ``tradutor de si
mesmo'': criando situações fincadas no contexto carioca do século \textsc{xix} e
inícios do \textsc{xx}, o escritor estabelece um paralelo entre elas e
referências ficcionais estrangeiras, de modo a provocar no leitor o
olhar crítico ante a própria realidade, junto com uma reflexão
relativizadora quanto a padrões locais e universais de conduta.

Contos que combinam uma representação crítica da sociedade, expressão
subjetiva de ilusões e uma construção singular em termos,
respectivamente, de concisão e de despertar comoção, ``Os brincos de
Sara'', de Alberto de Oliveira, e ``Tragédia dum capão de pintos'', de
Monteiro Lobato, foram selecionados por Graciliano Ramos para figurarem
na coletânea \emph{Contos e novelas}. Consagrado como poeta, Alberto de
Oliveira recebeu justa atenção de Graciliano para sua prosa, e o
referido conto de Lobato igualmente se destaca, com um caráter lúdico
próprio ao criador do Picapau Amarelo, junto com a dor das injustiças da
vida.

Por fim, quanto a Lima Barreto, se sua estreia em 1909,
\emph{Recordações do escrivão Isaías Caminha}, foi tida pela crítica dos
contemporâneos tão só como um romance \emph{à} \emph{clef}, dois anos
depois, \emph{Triste fim de Policarpo Quaresma} saiu em folhetim no
\emph{Jornal do Commercio} e, publicado em livro em 1915, teve recepção
elogiosa.

Contemporaneamente, a professora e pesquisadora Carmem Negreiros
ressalta, como tema importante para Lima Barreto, no cenário da Primeira
República, o debate sobre a educação como alavanca ao progresso: ``o
escritor sempre foi crítico do que se preconizava como `instrução
pública', aquela que propõe ensinar a ler, escrever e fazer contas,
apenas, sem qualquer premissa de conscientização. Ainda assim,
inacessível para grande parte da população''. E conclui, com precisão:
``Em suas múltiplas vertentes, a obra de Lima Barreto traz-nos as vozes
dos silenciados pela história cultural, com a linguagem que soube
incorporar as novas tecnologias, a riqueza da experiência urbana, o
diálogo tenso com a tradição literária''.

\chapter{Sobre o gênero}

Numa visão tradicional, o conto se caracteriza pela narração de único fato (a unidade dramática), em uma só ação, em espaço e tempo reduzidos (unidade espacial e temporal) e em um só tempo (unidade temporal). 

O conto e o romance são gêneros que apresentam, em comum, os componentes básicos da narrativa: 

•	o foco narrativo, isto é, o ponto de vista a partir do qual a história é contada, composto por um narrador;  
•	o enredo, ou seja, os fatos narrados; 
•	as personagens envolvidas nesses eventos, dentre as quais se destaca o protagonista; 
•	o tempo e o espaço em que a narrativa ocorre.   

A discussão sobre o que é conto enquanto gênero já rendeu muitas discussões. Mário de Andrade, ironizando esses debates intermináveis, numa tirada humorística tentou resolver a questão afirmando que “conto é tudo o que você quer chamar de conto”. A frase é interessante e faz rir, mas não soluciona o problema.

Em linhas bastante gerais podemos dizer que conto e romance se diferenciam pelo número de ações e de personagens, pela extensão e, de modo mais raro, pelo número de narradores. O conto apresenta um núcleo principal para o qual convergem as ações de todos os personagens, enquanto no romance podemos ter vários núcleos que se cruzam dentro de uma trama narrativa mais complexa. Por isso o conto apresenta um número reduzido de personagens, enquanto no romance podemos encontrar um número muito maior, às vezes chegando a várias dezenas. A maioria dos romances oscila entre 200 e 500 páginas, mas um conto tem algumas páginas apenas e, quando se aproxima de uma centena, como O Alienista, de Machado de Assis, já é considerado novela. Muitos contos, por sua vez, têm apenas uma ou duas páginas e são verdadeiras obras-primas, como ocorre com os do escritor paranaense Dalton Trevisan. A extensão, é, portanto, outro elemento que diferencia o conto do romance, abrigando, no meio do caminho, a novela.

Séculos antes de Cristo já se identifica o conto presente em obras como \textit{As mil e uma noites}. Na Idade Média, são históricos os \textit{Contos de Canterbury}, de Chaucer, considerado o pai da literatura inglesa, e o \textit{Decameron}, de Bocaccio, na Itália, em fins da Idade Média, que assinala a ruptura da moral medieval, através de enredos muitas vezes eróticos e debochados. Nos séculos XVI e XVII, são valorizados os contos de Cervantes (\textit{Histórias Exemplares}) e de Quevedo (\textit{A hora de todos}), os contos de Perrault e La Fontaine, com destaque para o francês considerado o arauto de iluminismo e grande defensor da liberdade de imprensa, Voltaire, cujo \textit{Cândido} foi publicado em 1758. 

Há diversos tipos de conto: realistas, psicológicos, fantásticos, maravilhosos, infantis, de humor, de fadas, históricos, míticos, moralizantes, ilustrativos, folclóricos, de terror e outros. É no romantismo e ao lado do romance, no século XIX, que o conto adquire a importância que tem até hoje. O russo Nicolas Gogol (1809-1852) e o francês Honoré de Balzac (1799-1850) são considerados os fundadores do conto moderno.

O chamado conto moderno muitas vezes não tem um final definido ou tem um final aberto, que fica a cargo de cada leitor, quando não se inicia pelo próprio final que acarreta uma retrospectiva, além de muitas indefinições e perguntas que cercam seus personagens e a condição humana atual, às voltas com inseguranças, incertezas e problemas de identidade, sem falar numa mais aguda consciência dos problemas sociais.

No Brasil, o conto surge da oralidade, nos primeiros tempos da
colonização, e muitos integram a chamada literatura de cordel.

O grande contista brasileiro, nosso maior escritor, é Joaquim Maria
Machado de Assis (1839--1908), da transição entre os séculos \textsc{xix} e \textsc{xx},
autor de uma lista formidável de contos, alguns aqui incluídos.

Além do volume de contos \emph{Insônia} (1947), com destaque para
``Minsk'' e ``Dois dedos'', Graciliano Ramos (1892--1953) escreveu
romances com base em contos, e capítulos de \emph{Vidas secas} (1938) e
de \emph{Infância} (1945) saíram primeiro na imprensa como contos. Tão
importante quanto Machado de Assis é João Guimarães Rosa (1908--1967),
com o antológico ``A terceira margem do rio'' (\emph{Primeiras
estórias}, 1962) e muitos outros, como ``O espelho'', que dialoga com o
título homônimo de Machado. E Clarice Lispector (1920--1977) com ``Feliz
aniversário'', ``Uma galinha'' (\emph{Laços de família}, 1960).

E o que falar de contos escritos por Dalton Trevisan, Murilo Rubião,
João Antonio, Marcos Rey, Moacyr Scliar, todos modernos e refazendo as
características do conto, alterando suas fronteiras, ampliando seus
limites, seus significados, dando espaço ao caos existencial que nos
define hoje?

\chapter{Atividades}

\section{Atividade 1}

\subsection{Conteúdo}

``Na Escola''

\subsection{Objetivos e justificativa}

Conhecer contos de grandes escritores brasileiros que escolheram como
cenário a sala de aula e teceram a representação da realidade social
brasileira, fincada em desigualdades, injustiças, violência e corrupção
desde sua origem colonial, junto com a expressão de desejos, frustrações
e conquistas das personagens, sejam alunos, professores, inspetores,
pais de alunos. A leitura desses contos abre o olhar para identificar
diferentes fatores de conflitos, suas bases históricas, sociais,
psicológicas, e para compreender o sentido de ética.

\subsection{Pré-leitura}

Pedir para os alunos pesquisarem sobre a vida e a obra de Medeiros e
Albuquerque e de Machado de Assis e promover uma aula dialogada a
respeito dos dois escritores. Contar que eles foram amigos e que,
conforme relata Medeiros, Machado visualizava com grande intensidade as
figuras literárias presentes nos livros. Vejam esta história curiosa:
Machado de Assis tinha lido ``As calças do Raposo'' na \emph{Revista
Brasileira}; dias depois, quando ia tomar o bonde no Largo da Carioca,
viu um homem, julgou que o conhecia, mas não se lembrava de onde; com um
esforço de memória, conseguiu identificar quem era: ``--- Ah! É o
Raposo, do Medeiros!''.

Indicar que os estudantes procurem os significados dos verbos
\emph{iludir}, \emph{colar} e \emph{delatar}, e dos substantivos
\emph{desilusão}, \emph{palmatória}, \emph{bolo}, \emph{delação} e
\emph{corrupção}. Apontar, em especial, que os sentidos de
\emph{iludir}, \emph{ilusão} e \emph{desilusão} estarão presentes em
todos os contos do livro, ressaltando que \emph{ilusão} se forma de
\emph{in-}, ``inclusão'', e \emph{ludere}, ``jogo, brincadeira''; e que
a palavra significava originalmente ironia e, depois, ``erro de
percepção'', ``engano da mente''.

Observar que o verbo \emph{colar} tem o substantivo homônimo, que
designa o objeto de adorno, a gargantilha, palavra originária do latim
\emph{collum} (pescoço). E então destacar as várias acepções de
``colar'', desde ``passar cola'' no sentido de ``grudar'', até ``copiar
as respostas alheias'', bem como o significado de informática, de
``copiar e colar, \emph{ctl c} e \emph{ctrl v}'', e também de ``colar
grau'', receber diploma superior. Em relação à palavra \emph{bolo},
enfatizar sua acepção nada doce, ligada ao vocábulo \emph{palmatória}.

Promover uma conversa inicial sobre questões éticas vinculadas à ação de
colar no sentido de trapacear: quem cola engana a si próprio, ao se
iludir com a boa nota sem deter o conhecimento para atingi-la; além
disso, engana o professor e os colegas, aproveitando-se do estudo
alheio.

Estimular os alunos a contarem experiências de colas, de cópias de
trabalhos de outros, e suas consequências, em termos de flagra e
punição, e de prejuízo na sequência da aprendizagem.

\subsection{Leitura}

(EF69LP53) Ler em voz alta textos literários diversos – como contos de amor, de humor, de suspense, de terror; crônicas líricas, humorísticas, críticas; bem como leituras orais capituladas (compartilhadas ou não com o professor) de livros de maior extensão, como romances, narrativas de enigma, narrativas de aventura, literatura infantojuvenil, – contar/recontar histórias tanto da tradição oral (causos, contos de esperteza, contos de animais, contos de amor, contos de encantamento, piadas, dentre outros) quanto da tradição literária escrita, expressando a compreensão e interpretação do texto por meio de uma leitura ou fala expressiva e fluente, que respeite o ritmo, as pausas, as hesitações, a entonação indicados tanto pela pontuação quanto por outros recursos gráfico-editoriais, como negritos, itálicos, caixa-alta, ilustrações etc., gravando essa leitura ou esse conto/reconto, seja para análise posterior, seja para produção de audiobooks de textos literários diversos ou de podcasts de leituras dramáticas com ou sem efeitos especiais e ler e/ou declamar poemas diversos, tanto de forma livre quanto de forma fixa (como quadras, sonetos, liras, haicais etc.), empregando os recursos linguísticos, paralinguísticos e cinésicos necessários aos efeitos de sentido pretendidos, como o ritmo e a entonação, o emprego de pausas e prolongamentos, o tom e o timbre vocais, bem como eventuais recursos de gestualidade e pantomima que convenham ao gênero poético e à situação de compartilhamento em questão.

Realizar a leitura de ``As calças do Raposo'' com os alunos, em voz
alta. Observar como o conto configura a caracterização do protagonista e
do seu filho por meio de seus sonhos, realizações e, sobretudo, das
frustrações que enfrentam em seu caminho. O inspetor Raposo é
apresentado como homem de cultura: foi político militante e jornalista
inteligente, porém tinha caído em ``desânimo inexplicável'', talvez
motivado por ``pequenos desgostos''. Em sua trajetória, sonhou ter um
filho formado, alcançou essa conquista, mas, devido a uma confusão
causada por um invejoso, sofreu a desilusão de não assistir à cerimônia
de formatura.

Verificar, junto com os estudantes, a forma como o conto de fato desenha
um personagem marcado por delicadeza única, que desperta a simpatia dos
leitores. Em especial, analisar a passagem em que o menino Fuinha delata
falsamente o Raposinho, provocando a generosidade e a revolta dos
colegas. Raposo se vê no impasse entre a seriedade de inspetor e o temor
de funcionário e pai amoroso, que não pode perder o emprego, nem ver o
filho fora da escola, nem vítima de injustiças. Refletir com os alunos
sobre o ato de delatar um inocente, injustiça que, no conto, mexe com os
sentimentos da classe, e leva a entender o sentido de ética.

Atentar para o episódio final, que dá título ao conto. Destacar que os
sonhos do Raposo e de seu filho seguem um caminho ético, de estudo e
respeito. Já o menino Fuinha, invejoso, temia espiões de seus atos
maldosos e tenta macular a vitória dos Raposos, conseguindo isso só no
dia da formatura. Refletir sobre os efeitos desse episódio no conto,
ressaltando que, de todo modo, o Raposinho se torna professor, com
dignidade.

Ler com os alunos o ``Conto de escola'', de Machado de Assis. Notar que
aqui também há um menino delator, mas os colegas por ele acusados de
fato agiram de má-fé: Raimundo, filho do professor, ofereceu uma moeda
para Pilar, em troca de explicações da matéria. Analisar, com os
estudantes, como a caracterização dos meninos inclui os sonhos deles
(ser livre; saber a lição só para aplacar a ira do pai-professor) e as
expectativas que os pais projetam neles (serem bem-sucedidos), de forma
que a ação do conto advém do choque entre os desejos das crianças, as
aspirações dos pais e os limites de todos, ressaltando a representação
de uma realidade marcada por violência, corrupção e delação. Ao mesmo
tempo, vale atentar para imagens como esta, projeções de um horizonte de
liberdade: ``E lá fora, no céu azul, por cima do morro, o mesmo eterno
papagaio, guinando a um lado e outro, como se me chamasse a ir ter com
ele''.

\subsection{Pós-leitura}

Estabelecendo um vínculo de interdisciplinaridade com as áreas de
História e Filosofia, propor que os estudantes façam pesquisas sobre as
seguintes questões suscitadas pelos contos: o uso da palmatória no
Brasil, vinculada à origem escravocrata do país; fatos históricos
envolvendo delação e corrupção. Nesse sentido, recomendar que leiam ``D.
Maria'', ``Adelaide'', ``Um novo professor'' e ``A criança infeliz'',
capítulos de \emph{Infância}, de Graciliano Ramos, \emph{Da senzala à
colônia}, de Emília Viotti da Costa, o \emph{Dicionário da escravidão e
liberdade: 50 textos críticos}, organizado por Lilia Moritz Schwarcz e
Flávio dos Santos Gomes, além dos demais contos dessa seção, ``Um bom
diretor'', de Lima Barreto, e ``Pé no chão, Vidinha ociosa'', de
Monteiro Lobato. \textbf{(EF08HI14)} Discutir a noção da tutela dos
grupos indígenas e a participação dos negros na sociedade brasileira do
final do período colonial, identificando permanências na forma de
preconceitos, estereótipos e violências sobre as populações indígenas e
negras no Brasil e nas Américas. \textbf{(EF09HI01)} Descrever e
contextualizar os principais aspectos sociais, culturais, econômicos e
políticos da emergência da República no Brasil; \textbf{(EF09HI03)}
Identificar os mecanismos de inserção dos negros na sociedade brasileira
pós-abolição e avaliar os seus resultados; \textbf{(EF09HI04)} Discutir
a importância da participação da população negra na formação econômica,
política e social do Brasil.

Promover debates a respeito dos temas ``Educação e violência'',
``Corrupção e ética''. \textbf{(EF09HI23)} Identificar direitos civis,
políticos e sociais expressos na Constituição de 1988 e relacioná-los à
noção de cidadania e ao pacto da sociedade brasileira de combate a
diversas formas de preconceito, como o racismo.

Propor que, em grupos, os alunos escolham um dos contos dessa primeira
seção e preparem um roteiro a partir deles, para ensaiarem e o
encenarem, como teatro. Podem fazer adaptações nas histórias, incluir
outras personagens, criar figurinos, sonoplastia. 

Sugerir que escrevam textos ficcionais ou dissertativos sobre situações
de cola no ambiente escolar. 

\subsection{Tempo estimado}

Seis aulas de 50 minutos.

\section{Atividade 2}

\subsection{Conteúdo}

``Na Realidade Social''

\subsection{Objetivos e justificativa}

Levar os estudantes a perceberem a habilidade de escritores de prenderem
a atenção do leitor por meio da construção de situações cômicas ou
trágicas, nas quais há elementos a princípio invisíveis para as
personagens. Uns despertando comoção, outros provocando riso, esses
contos possibilitam aos estudantes desenvolverem a consciência crítica
quanto a motivos e pessoas a serem vistos na realidade. A compaixão
instigada no leitor por um conto como ``O fisco'' é fundamento da moral
e de força poética.

\subsection{Pré-leitura}

Pedir para os alunos pesquisarem sobre a vida e a obra de Monteiro
Lobato e de Lima Barreto e promover uma aula dialogada a respeito dos
dois escritores.

Apresentar os títulos dos contos da segunda seção, ``Na realidade
social'': ``Dona Expedita'', ``O fisco (Conto de Natal)'' e ``Os
pequeninos'', de Monteiro Lobato, ``Galeria póstuma'', de Machado de
Assis, e ``O homem que sabia javanês'', de Lima Barreto. Abrir um
bate-papo, pedindo que os alunos imaginem o que pode ser a história de
cada um, e lembrar que todas essas narrativas configuram ilusões
seguidas de confusões. Para esse objetivo, é interessante pedir que
busquem os significados das palavras \emph{expedito}, \emph{fisco},
\emph{galeria}, \emph{póstumo} e \emph{javanês}, também de
\emph{diligente}, \emph{receita}, \emph{fazenda}, \emph{erário},
\emph{imposto}. Vale assinalar a materialidade da origem das palavras:
conforme o dicionário \emph{Houaiss}, fisco, do latim \emph{fiscus}, era
o ``saco de guardar dinheiro; as rendas do príncipe, providas pelo
erário público''. E também salientar os sentidos concretos e figurados
das palavras, por exemplo de \emph{galeria}.

Perguntar em especial se sabem o significado de \emph{quiproquó} e
explicar que vem da expressão latina \emph{quid pro quo}, ``uma coisa
pela outra''. Anunciar então que, no conto ``Dona Expedita'', o engano
de tomar uma coisa por outra provoca riso, embora tenha um fundo
trágico, que é atualíssimo, a situação de desemprego e desigualdade
social. E incitar a curiosidade para descobrirem se a protagonista era
mesmo uma pessoa expedita.

Dividir a turma em grupos, para se dedicarem à leitura de cada um dos
contos.

\subsection{Leitura}

(EF69LP53) Ler em voz alta textos literários diversos – como contos de amor, de humor, de suspense, de terror; crônicas líricas, humorísticas, críticas; bem como leituras orais capituladas (compartilhadas ou não com o professor) de livros de maior extensão, como romances, narrativas de enigma, narrativas de aventura, literatura infantojuvenil, – contar/recontar histórias tanto da tradição oral (causos, contos de esperteza, contos de animais, contos de amor, contos de encantamento, piadas, dentre outros) quanto da tradição literária escrita, expressando a compreensão e interpretação do texto por meio de uma leitura ou fala expressiva e fluente, que respeite o ritmo, as pausas, as hesitações, a entonação indicados tanto pela pontuação quanto por outros recursos gráfico-editoriais, como negritos, itálicos, caixa-alta, ilustrações etc., gravando essa leitura ou esse conto/reconto, seja para análise posterior, seja para produção de audiobooks de textos literários diversos ou de podcasts de leituras dramáticas com ou sem efeitos especiais e ler e/ou declamar poemas diversos, tanto de forma livre quanto de forma fixa (como quadras, sonetos, liras, haicais etc.), empregando os recursos linguísticos, paralinguísticos e cinésicos necessários aos efeitos de sentido pretendidos, como o ritmo e a entonação, o emprego de pausas e prolongamentos, o tom e o timbre vocais, bem como eventuais recursos de gestualidade e pantomima que convenham ao gênero poético e à situação de compartilhamento em questão.

Pedir que os alunos se organizem em grupos, para lerem os contos da
seção ``Na realidade social''. Chamar a atenção para o fato de que esses
textos se constroem com base em confusões, provocadas por enganos de
expectativas no plano das relações sociais. Destacar que, como resultado
dessas ilusões, alguns textos são marcados por humor, como ``Galeria
póstuma'', de Machado de Assis, ``Dona Expedita'', de Monteiro Lobato, e
``O homem que sabia javanês'', de Lima Barreto. Já em outros, como ``O
fisco (Conto de Natal)'', de Monteiro Lobato, predomina um sentido
trágico. E ``Os pequeninos'' surpreende com sua comicidade junto com
lições de biologia.

Quanto à ``Galeria póstuma'' criada por Machado de Assis, notar como
revela, com humor, o jogo de aparências que marca as relações em
sociedade. Verificar com os alunos a relativização do olhar em relação
às pessoas, que podem ser consideradas sob ângulos diversos. Imagens
idealizadas das pessoas, nunca inteiramente condizentes com a realidade,
geram ilusões, mal-entendidos e frustrações.

O conto tem início com a morte do deputado Joaquim Fidélis e a tristeza
do seu círculo de conhecidos e de cinco familiares. E na sequência nos
possibilita acompanhar o tragicômico impasse vivido por seu sobrinho
Benjamim: ele fica feliz ao encontrar, diante dos familiares
aparentemente consternados com aquela morte, um diário do finado; mas
logo vem a perplexidade de descobrir, nas páginas do diário, o olhar
irônico com que o tio criticava essas pessoas, que lhe deviam favores.
Vale destacar que Benjamin conserta, no morto, o ``leve arregaço irônico
ao canto esquerdo da boca'' e, depois, precisa ocultar dos parentes a
mordacidade com que, por trás dos modos cordiais, o tio via o caráter
tolo e interesseiro deles. A descrição de Galdino Madeira, por exemplo,
nos desperta o riso: ``O melhor coração do mundo e um caráter sem
mácula; mas as qualidades do espírito destroem as outras. Emprestei-lhe
algum dinheiro, por motivo da família, e porque me não fazia falta. Há
no cérebro dele um certo furo, por onde o espírito escorrega e cai no
vácuo. Não reflete três minutos seguidos''. E o riso se prolonga na
ironia final, que recai sobre Benjamin: ele não divulga o teor do
diário, para proteger os familiares de se desiludirem com Joaquim
Fidélis, agora fantasma sarcástico; porém, o resultado é que o
consideram um herdeiro soberbo, muito diferente do tio, tão amável.

Acompanhar nos grupos a leitura dos demais contos dessa seção,
ressaltando que os quiproquós, os caminhos entre ilusões e confusões,
ocorrem em torno da necessidade de ter e conservar um emprego, uma fonte
de sustento. Indicar que os estudantes anotem observações a respeito da
construção dos contos: como se formam a ambiguidade, o suspense, os
efeitos de riso e de comoção.

Apontar o fato de que as ilusões e desenganos envolvem elementos a
princípio invisíveis para as personagens: Dona Expedita se imagina
prestes a conseguir um ótimo emprego, pois ignora quem era sua
interlocutora; o pequeno Pedrinho, de ``O fisco (Conto de Natal)'',
sonha em ser engraxate para ajudar a família, mas não sabe que proíbem
crianças de trabalhar; Benjamim não supõe que o tio ironizava os
familiares; Manuel, de ``Os pequeninos'', embora se saiba inocente,
desconhece a causa do desaparecimento do saco de arroz; o aprendiz de
javanês não poderia supor os rumos de sua maestria.

Chamar a atenção para a forma como os contos configuram os olhares
diferentes de patrões, proprietários e de empregados, dependentes,
deixando ver questões como o desemprego, a desigualdade social. 

\subsection{Pós-leitura}

Sugerir que os alunos se dividam em grupos para escreverem o roteiro de
um quiproquó como o de ``Dona Expedita''; pode ter por base uma
entrevista de emprego, um encontro amoroso combinado pela internet.
Então, organizam-se para gravar um vídeo ou um áudio como radionovela.

Propor que os estudantes leiam o capítulo ``Os astrônomos'', de
\emph{Infância}, de Graciliano Ramos, e debater com eles sobre métodos
de estudo de línguas (como fez o homem que sabia javanês) e de outras
matérias de maneira autodidata, partindo do conhecimento existente e se
debruçando sobre o não sabido.

Com base em ``Os pequeninos'', propor aos alunos que façam uma pesquisa
sobre relações ecológicas de predação, também de mutualismo, sobre a
vida em sociedade das formigas, o bacilo de Koch. Depois, a tarefa é
prepararem cartazes e gravarem conversas em \emph{podcasts} sobre as
relações entre animais, o homem como ``o terror da bicharia toda'' e a
ação de micro-organismos, considerando inclusive a pandemia do
coronavírus. \textbf{(EF09CI11)} Discutir
a evolução e a diversidade das espécies com base na atuação da seleção
natural sobre as variantes de uma mesma espécie, resultantes de processo
reprodutivo.

\subsection{Tempo estimado} Seis aulas de 50 minutos.

\section{Atividade 3}

\subsection{Conteúdo} ``Nas Relações Amorosas''

\subsection{Objetivos e justificativa}

Despertar a sensibilidade e a consciência dos estudantes para a
construção do ponto de vista narrativo e seus efeitos nos textos. E,
assim, destacar o sentido de relativização, marcante nos caminhos cheios
de reversibilidade entre ilusão e desilusão, entre a expectativa de
benefícios e o sofrimento de infortúnios, dos contos aqui presentes e da
realidade. O aprendizado quanto à relativização inclui um horizonte de
compreensão e respeito ao outro, de ética.

\subsection{Pré-leitura}

Solicitar que os estudantes pesquisem a respeito de Alberto de Oliveira
e de Artur Azevedo.

Recordar com os alunos o sentido da palavra \emph{antologia}, que tem
por sinônimo florilégio: etimologicamente, ``ação de colher flores,
coleção de trechos literários''. E observar que o organizador de uma
coletânea deve definir critérios para a escolha das obras: temáticos,
estéticos, cronológicos. Relatar aos alunos que um dos motivos para a
inclusão de ``Primas de Sapucaia!'' nesta antologia é a predileção do
próprio escritor por esse texto. Segundo Max Fleiuss, quando preparou
para \emph{A Semana} uma seleção dos contos publicados no
periódico, consultou Machado de Assis, que imediatamente indicou tal
conto em ``carinhoso bilhete''.

Anunciar aos estudantes que outro critério aqui evidentemente é o
temático: todos os contos desta seção trazem ilusões e desilusões no
plano das relações amorosas. E enfatizar que a força de concisão é uma
virtude estética de contos: a capacidade de constituírem, em poucas
páginas, situações significativas em termos de representação social
crítica e de expressão de impasses subjetivos. Nesse sentido, apontar
que ``Os brincos de Sara'', de Alberto de Oliveira, e ``Útil inda
brincando'', de Artur Azevedo, foram também selecionados pelo escritor
Graciliano Ramos para a antologia de contos das várias regiões do país
que preparou nos anos 1940 para a Casa do Estudante do Brasil. Depois de
pesquisar durante dois meses na Academia Brasileira de Letras e outros
dois na Biblioteca Nacional e de enviar cartas a academias de letras do
país e a diversos intelectuais pedindo narrativas, ele escolheu cem
contos e os organizou em três volumes: I, Norte e Nordeste; \textsc{ii}, Leste;
\textsc{iii}, Sul e Centro-Oeste. Incluiu escritores consagrados como Machado de
Assis, Lima Barreto, Mário de Andrade, Carlos Drummond de Andrade,
Fernando Sabino e alguns pouco conhecidos hoje. Sendo a edição de
\emph{Contos e novelas} de 1957, posterior à morte de Graciliano, seu
amigo Aurélio Buarque de Holanda acrescentou a ela o conto ``Minsk'',
antes publicado em \emph{Insônia} (1947).

Pedir uma pesquisa a respeito da frase ``Útil inda brincando'', que
intitula o conto de Artur Azevedo. Curiosamente, era a legenda de uma
escultura no chafariz dos Jacarés, a Fonte do Menino, no Passeio Público
do Rio de Janeiro. Refletir sobre o significado dessa combinação de ser
útil e agradável.

Solicitar que os alunos pesquisem sobre a fonte shakespeariana da
abertura de ``A cartomante'' e conjecturem acerca de significados
possíveis: ``Hamlet observa a Horácio que há mais coisas no céu e na
terra do que sonha a nossa filosofia''. 

\subsection{Leitura}

(EF69LP53) Ler em voz alta textos literários diversos – como contos de amor, de humor, de suspense, de terror; crônicas líricas, humorísticas, críticas; bem como leituras orais capituladas (compartilhadas ou não com o professor) de livros de maior extensão, como romances, narrativas de enigma, narrativas de aventura, literatura infantojuvenil, – contar/recontar histórias tanto da tradição oral (causos, contos de esperteza, contos de animais, contos de amor, contos de encantamento, piadas, dentre outros) quanto da tradição literária escrita, expressando a compreensão e interpretação do texto por meio de uma leitura ou fala expressiva e fluente, que respeite o ritmo, as pausas, as hesitações, a entonação indicados tanto pela pontuação quanto por outros recursos gráfico-editoriais, como negritos, itálicos, caixa-alta, ilustrações etc., gravando essa leitura ou esse conto/reconto, seja para análise posterior, seja para produção de audiobooks de textos literários diversos ou de podcasts de leituras dramáticas com ou sem efeitos especiais e ler e/ou declamar poemas diversos, tanto de forma livre quanto de forma fixa (como quadras, sonetos, liras, haicais etc.), empregando os recursos linguísticos, paralinguísticos e cinésicos necessários aos efeitos de sentido pretendidos, como o ritmo e a entonação, o emprego de pausas e prolongamentos, o tom e o timbre vocais, bem como eventuais recursos de gestualidade e pantomima que convenham ao gênero poético e à situação de compartilhamento em questão.

Em ``Primas de Sapucaia!'', chamar a atenção para o sentido inicial das
primas como metonímia para as pessoas inoportunas, que surgem no exato
momento em que alguém estava prestes a se aproximar de um objeto de
desejo. No conto, obrigado a ciceronear as primas, o narrador sofre o
impedimento de se aproximar de uma mulher desejada, Adriana, que, como
Capitu, possui olhos que arrastam e devastam, mas não lhe conhecemos a
voz. Nesse sentido, apontar a questão do ponto de vista do narrador, que
evidentemente tem decorrências na construção e na interpretação dos
textos.

Destacar que, na sequência do conto, sobrevém o princípio machadiano de
relativização de tudo, que causa impacto nos leitores: o narrador se
regozija diante da ruína de um homem, decorrente do envolvimento com
aquela moça; então as primas, que pareceram terrível obstáculo, se
mostram úteis, ``antes um benefício do que um infortúnio''.

Além de provocar os alunos a refletirem sobre essa relativização entre
malefícios e benefícios, é preciso ressaltar a imagem concebida por
Machado de Assis que vincula amor e livro --- sem índice, sem marginália
e sem fita-marcador ---, no momento em que o protagonista vive a ilusão
de plenitude ao lado da mulher, mesmo considerando que por fim virá o
desencanto:

\begin{quote}
{[}\ldots{}{]} Supusemo-nos estrangeiros, e realmente não éramos outra
coisa; falávamos uma língua que nunca ninguém antes falara nem ouvira.
Os outros amores eram, desde séculos, verdadeiras contrafações; nós
dávamos a edição autêntica. Pela primeira vez, imprimia-se o manuscrito
divino, um grosso volume que nós dividíamos em tantos capítulos e
parágrafos quantas eram as horas do dia ou os dias da semana. O estilo
era tecido de sol e música; a linguagem compunha-se da fina flor dos
outros vocabulários. Tudo o que neles existia, meigo ou vibrante, foi
extraído pelo autor para formar esse livro único --- livro sem índice,
porque era infinito --- sem margens, para que o fastio não viesse
escrever nelas as suas notas --- sem fita, porque já não tínhamos
precisão de interromper a leitura e marcar a página.
\end{quote}

No conto ``Os brincos de Sara'', observar como são sucintas e fortes, em
seu teor crítico, as descrições do político estúpido e da presença
discreta dos criados em meio a uma festa da elite: ``--- Vocês o
conhecem, raso como uma calçada! Formou-se, é verdade, é doutor, doutor
na asneira, como já ouvi dizer de um. Incapaz de sustentar uma
discussão, incapaz de abrir a boca que não diga tolice, que irá ele
fazer na Câmara, na hipótese de ser eleito?'' / ``Caras de criados
irrompiam do corredor, espiando''.

Sobretudo vale atentar para a forma como, nesse conto e também em ``Útil
inda brincando'' e em ``A cartomante'', os autores investem na
construção do suspense, por meio de imagens, situações e personagens
marcantes de relações amorosas adúlteras, e de um desfecho
surpreendente. Vejam-se: a imagem dos brincos para falar do adultério; a
oferta infame de utilidade e prazer, ao partilhar a amante, sobrelevada
por remorso apaixonado; o contraste entre a figura ardilosa da
cartomante e a credulidade do casal.

\subsection{Pós-leitura}

Em articulação interdisciplinar com as aulas de História, solicitar que
os alunos pesquisem sobre as leis antigas referentes ao adultério, que
recaíam em especial sobre a mulher. E sobre o contexto atual dos
feminicídios e a lei Maria da Penha. \textbf{(EF89LP17)} Relacionar textos e documentos legais e normativos de importância universal, nacional ou local que envolvam direitos, em especial, de crianças, adolescentes e jovens – tais como a Declaração dos Direitos Humanos, a Constituição Brasileira, o ECA -, e a regulamentação da organização escolar – por exemplo, regimento escolar -, a seus contextos de produção, reconhecendo e analisando possíveis motivações, finalidades e sua vinculação com experiências humanas e fatos históricos e sociais, como forma de ampliar a compreensão dos direitos e deveres, de fomentar os princípios democráticos e uma atuação pautada pela ética da responsabilidade (o outro tem direito a uma vida digna tanto quanto eu tenho). 

Outro motivo de pesquisa é a história do livro, da marginália e dos
marcadores de página.

Propor que os alunos assistam aos dois filmes inspirados no conto ``A
cartomante'': o de 1974, dirigido por Marcos Faria, e o de 2004,
dirigido por Wagner de Assis e Pablo Uranga. Pedir que escrevam uma
resenha, apontando diferenças e semelhanças em relação ao texto de
Machado de Assis, e destacando passagens das obras de que gostaram e que
não apreciaram. Depois, organizam-se em grupos e gravam \emph{podcasts},
reunindo trechos de suas várias leituras críticas. 

Promover um debate, levando a pensar sobre os temas e a construção dos
contos e dos filmes, lançando perguntas assim: quais as possíveis formas
de retratar o adultério, qual o papel da cartomante? Considerando ``Útil
inda brincando'', imaginar o efeito, para o público da época, de o conto
retomar, no contexto de relações amorosas, essa expressão conhecida, que
era legenda de um chafariz justamente lúdico e útil (fonte gratuita de
água). E, tendo em mente sobretudo o público de hoje, apontar
criticamente para o mal de tratar o outro como objeto. Questionar em que
medida o remorso do protagonista e a viuvez de Clotilde aplacam o
aspecto machista do texto. Assim, refletir com os alunos sobre a
necessidade de respeito em todas as relações.

Com base nas leituras e no debate, propor que os alunos escolham um dos
contos e o reescrevam mudando o foco narrativo, talvez criando finais
diferentes, ou redijam contos inspirados nos desta seção. Por exemplo:
como Adriana de ``Primas de Sapucaia!'', um criado da festa de ``Os
brincos de Sara'', a cartomante do conto de Machado de Assis ou
Clotilde, de ``Útil inda brincando'', contariam as histórias?

Indicar que os estudantes leiam os textos dos colegas, sugiram possíveis
alterações e, em grupos, organizem coletâneas de seus contos.

\section{Atividade 4}

\subsection{Conteúdo} ``Na Língua''e ``Entre Animais''

\subsection{Objetivos e justificativa}

Mostrar a necessidade de ter maior conhecimento das palavras e dos fatos
e de compreender os limites e alcances da percepção da realidade.
Despertar a imaginação, a sensibilidade e o respeito para com os outros,
sejam animais ou pessoas, com suas características diferentes.

\subsection{Pré-leitura}

Perguntar aos alunos sobre a prática de procurar o significado de
palavras no dicionário. Conhecem a forma como se buscam os verbetes,
sejam nomes ou verbos? Falar sobre a ordem alfabética, a descrição
gramatical dos verbetes, levar dicionários em volumes e minidicionários
para os estudantes se divertirem com o exercício de procurar palavras.

Recordando o sentido de \emph{quiproquó}, anunciar que lerão uma
história cheia de quiproquós de palavras com um personagem asneirento.
Depois, uma narrativa tendo por cenário o ambiente familiar, com pais e
filhos. E outras duas envolvendo animais, em especial aves. Nesse
sentido perguntar aos alunos se sabem o que é \emph{ornitologia} e
buscar com eles a palavra no dicionário.

\subsection{Leitura}

(EF69LP53) Ler em voz alta textos literários diversos – como contos de amor, de humor, de suspense, de terror; crônicas líricas, humorísticas, críticas; bem como leituras orais capituladas (compartilhadas ou não com o professor) de livros de maior extensão, como romances, narrativas de enigma, narrativas de aventura, literatura infantojuvenil, – contar/recontar histórias tanto da tradição oral (causos, contos de esperteza, contos de animais, contos de amor, contos de encantamento, piadas, dentre outros) quanto da tradição literária escrita, expressando a compreensão e interpretação do texto por meio de uma leitura ou fala expressiva e fluente, que respeite o ritmo, as pausas, as hesitações, a entonação indicados tanto pela pontuação quanto por outros recursos gráfico-editoriais, como negritos, itálicos, caixa-alta, ilustrações etc., gravando essa leitura ou esse conto/reconto, seja para análise posterior, seja para produção de audiobooks de textos literários diversos ou de podcasts de leituras dramáticas com ou sem efeitos especiais e ler e/ou declamar poemas diversos, tanto de forma livre quanto de forma fixa (como quadras, sonetos, liras, haicais etc.), empregando os recursos linguísticos, paralinguísticos e cinésicos necessários aos efeitos de sentido pretendidos, como o ritmo e a entonação, o emprego de pausas e prolongamentos, o tom e o timbre vocais, bem como eventuais recursos de gestualidade e pantomima que convenham ao gênero poético e à situação de compartilhamento em questão.

Ao ler ``Plebiscito'' e ``As asneiras do Guedes'', de Artur Azevedo,
identificar, junto com os alunos, a forma como o humor recai sobre o
desconhecimento de palavras e a tentativa de parecer sábio,
ilusoriamente, para os outros. Em ``Plebiscito'', com poucas palavras, o
escritor nos provoca o riso, ao presentificar o vergonhoso disfarce de
ignorância de um pai ante uma pergunta do filho: ``--- Papai, que é
plebiscito?/ O senhor Rodrigues fecha os olhos imediatamente para fingir
que dorme''. Realçar para os estudantes como é simples e nada vergonhoso
recorrer ao dicionário.

Ao analisar a construção do sentido de ``As asneiras do Guedes'',
assinalar que os quiproquós e o riso resultam da proximidade sonora de
palavras como adúltera/adulta, sensual/senso, junto com a caracterização
de tolo e convencido do protagonista, ``o maior asneirão que o sol
cobre''. Apontar que \emph{coabitar} incluía hífen e h, e explicar que
as palavras sofrem mudanças gráficas e semânticas ao longo do tempo.

Pedir para os alunos observarem como o narrador configura a singela
história do amor do galo Peva pelo trio pintinho, peruzinho e
marrequinho, os filhotes órfãos, a qual termina em revolta e tragédia,
dada a indiferença dos homens e da própria natureza por aquele afeto
puro. E que notem a forma criativa como Lobato traduz o olhar dos
animais em relação aos homens, que inclui desconfiança, incompreensão,
submissão e medo. Assim, vão perceber a crítica à falta de atenção dos
homens, cegos às manifestações de afeto dos animais e à dor que lhes
causam. Num mundo imediatista, de primazia de interesses individuais e
consumismo, é fundamental mostrar aos alunos a disponibilidade afetiva
do galo Peva, que adota as aves órfãs e sofre enormemente por elas.

Durante a leitura de ``Ideias de canário'', enfatizar a força de
concisão de frases como ``Todo eu era canário'', que sinaliza a entrega
do comprador do passarinho à intenção de compreender a linguagem da ave.
E observar a beleza de imagens como ``O mundo --- concluiu solenemente
--- é um espaço infinito e azul, com o sol por cima'', que marca a
plenitude da liberdade do passarinho. Mas sobretudo perceber a ironia em
relação aos limites da nossa consciência, circunscrita sempre ao mundo
que nos cerca, conforme se vê na pergunta do canário, esquecido do
passado na gaiola: ``Mas há mesmo lojas de belchior?''.

\subsection{Pós-leitura}

Compreender o sentido cômico da referência final à palavra
\emph{plebiscito} como estrangeirismo. Pedir uma pesquisa a respeito de
estrangeirismo: os alunos buscam exemplos de estrangeirismos e refletem
sobre seu vínculo com inovações tecnológicas, sua incorporação na língua
e na origem de neologismos. Lembrar termos aportuguesados, como
\emph{futebol}, \emph{sanduíche}, \emph{espaguete}, e os não
aportuguesados, alguns hoje muito usados, como \emph{delivery},
\emph{drive-thru}, \emph{designer}, \emph{link}, \emph{site},
\emph{on-line}. \textbf{(EF09LP12)} Identificar estrangeirismos,
caracterizando-os segundo a conservação, ou não, de sua forma gráfica de
origem, avaliando a pertinência, ou não, de seu uso.

Para estudo na disciplina de História, vale propor a pesquisa do
significado de plebiscito na Roma antiga e atualmente.

Para estudo na disciplina de Ciências, pedir que pesquisem as
semelhanças e diferenças entre as aves-personagens do conto de Monteiro
Lobato: galo, peru, marreco, urubu. \textbf{(EF09CI11)} Discutir a
evolução e a diversidade das espécies com base na atuação da seleção
natural sobre as variantes de uma mesma espécie, resultantes de processo
reprodutivo.

Pedir que, em grupos, os alunos escolham um dos contos de Artur Azevedo
e preparem sua representação teatral, atentando para a caracterização
das personagens e para o efeito de riso provocado pelos diálogos.

Com base em ``Tragédia dum capão de pintos'', realizar um debate sobre a
relação entre os homens e os animais e sua configuração literária. Nesse
sentido, vale retomar os textos ``Baleia'' e ``Minsk'', de Graciliano
Ramos, que também figuram poeticamente a morte de animais. Esse debate
propiciado pela história do galo Peva e seus filhotes tão queridos pode
incluir questões importantes como a adoção e o veganismo. Incluir na
conversa o sentido de relativização, de compreensão dos limites da
realidade e das perspectivas, o qual ressalta de ``Ideias de canário''.
\textbf{(EF69LP15)} Apresentar argumentos e contra-argumentos coerentes,
respeitando os turnos de fala, na participação em discussões sobre temas
controversos e/ou polêmicos.

Sugerir que escrevam textos ficcionais ou dissertativos sobre as
relações entre homens e animais, incluindo diálogos com animais e as
questões da adoção e do veganismo. 

\chapter{Sugestões de referências complementares}

\begin{itemize}
\item Academia Brasileira de Letras:
\url{https://www.academia.org.br/academicos/membros}.

\item Acervo \textsc{fgv-cpdoc}: \url{http://www.fgv.br/cpdoc/acervo/}.

\item \textsc{albuquerque}, Medeiros e. \emph{Homens e cousas da Academia Brasileira}.
Rio de Janeiro: Renascença, 1934.

\item \_\_\_\_\_\_. ``Machado de Assis: \emph{Esaú e Jacob} (Garnier
Editor)''. \emph{A Notícia}, ``Crônica Literária'', Rio de Janeiro, 30
set. 1904 e 1º out. 1904, p. 3.

\item \_\_\_\_\_\_. ``Machado de Assis: \emph{Relíquias de casa velha}''.
\emph{A Notícia}, Rio de Janeiro, 23 e 24 fev. 1906.

\item \_\_\_\_\_\_. \emph{Quando eu era vivo. Memórias. 1867 a 1934}. Edição
póstuma e definitiva. Rio de Janeiro: Record, 1981.

\item \textsc{alves}, Ieda Maria. ``A integração dos neologismos por empréstimo ao
léxico português''. \emph{Alfa: Revista de Linguística}, São Paulo, vol.
28, n. 1, 2001. Disponível em:
\url{https://periodicos.fclar.unesp.br/alfa/article/view/3681}.

\item \textsc{bosi}, Alfredo. \emph{História concisa da literatura brasileira}. São
Paulo: Cultrix, 1975.

\item \_\_\_\_\_\_. ``O enigma do olhar''. In: \emph{Machado de Assis: o
enigma do olhar}. São Paulo: Ática, 1999.

\item \textsc{candido}, Antonio. ``Esquema de Machado de Assis''. In: \emph{Vários
escritos}. 3. ed. rev. e ampl. São Paulo: Duas Cidades, 1995.

\item \textsc{cavalheiro}, Edgard. \emph{Monteiro Lobato, vida e obra}. 2 tomos. São
Paulo: Companhia Editora Nacional, 1955.

\item \textsc{costa}, Emília Viotti da. \emph{Da senzala à colônia}. São Paulo: Editora
Unesp, {[}1997{]}, 2016.

\item\emph{\textsc{dicionário} Aulete digital}: \url{http://www.aulete.com.br/}.

\item \textsc{guimarães}, Hélio de Seixas \& \textsc{lebensztayn}, Ieda (orgs.). \emph{Escritor
por escritor: Machado de Assis segundo seus pares (1908--1939;
1939--2008)}, 2 vols. São Paulo: Imprensa Oficial do Estado de São Paulo,
2019.

\item \textsc{houaiss}, Antonio \& \textsc{villar}, Mauro de Salles. \emph{Dicionário Houaiss da
língua portuguesa}. Rio de Janeiro: Objetiva, 2001.

\item \textsc{lobato}, Monteiro \& \textsc{rangel}, Godofredo. \emph{A barca de Gleyre: quarenta
anos de correspondência literária}. São Paulo: Brasiliense, 1968.

\item\url{https://machadodeassis.net/}. \emph{Referências na Ficção
Machadiana}.

\item \textsc{negreiros}, Carmem. \emph{Lima Barreto em quatro tempos}. Belo Horizonte:
Relicário, 2019.

\item \_\_\_\_\_\_. ``Múltiplas faces de Lima Barreto''. \emph{A Terra é
Redonda}, 13 maio 2021. Disponível em:
\url{https://aterraeredonda.com.br/multiplas-faces-de-lima-barreto/}.

\item \textsc{pereira}, Lúcia Miguel. \emph{Machado de Assis: estudo crítico e
biográfico}. Belo Horizonte; São Paulo: Itatiaia; Edusp, {[}1936{]}
1988.

\item \textsc{prado}, Natália Cristine. ``Estrangeirismos''. In: \emph{O uso do inglês
em contexto comercial no Brasil e em Portugal: questões linguísticas e
culturais}. São Paulo: Editora Unesp; São Paulo: Cultura Acadêmica,
2015, pp. 27--70. Disponível em:
\url{https://books.scielo.org/id/rxwst/pdf/prado-9788579836541-03.pdf}.

\item \textsc{ramos}, Graciliano. ``Baleia''. In: \emph{Vidas secas}. 59 ed. Rio de
Janeiro: Record, {[}1938{]} 1989.

\item \textsc{ramos}, Graciliano. \emph{Infância}. 50 ed. Rio de Janeiro: Record,
{[}1945{]} 2020.

\item \textsc{ramos}, Graciliano. ``Minsk''. In: \emph{Insônia}. {[}1947{]}. 32. ed.
Rio de Janeiro: Record, 2017.

\item \textsc{ramos}, Graciliano. \emph{Minsk}. Ilustrações de Rosinha. Rio de Janeiro:
Galera Record, 2013.

\item \textsc{schwarcz}, Lilia Moritz \& \textsc{gomes}, Flávio dos Santos. \emph{Dicionário da
escravidão e liberdade: 50 textos críticos}. São Paulo: Companhia das
Letras, 2018.

\item \textsc{villaça}, Alcides. ``Machado de Assis, tradutor de si mesmo''.
\emph{Revista Novos Estudos Cebrap}. São Paulo, 1998, n. 51, pp. 3--14.
\end{itemize}

\section{Filmes, adaptações do conto de Machado de Assis}

\begin{itemize}
\item\emph{A cartomante} (1974). Direção de Marcos Faria; roteiro de Faria,
Salim Miguel e Eglê Malheiros.

\item\emph{A cartomante} (2004). Direção de Wagner de Assis e Pablo Uranga;
roteiro de Wagner de Assis.
\end{itemize}

\section{Bibliografia comentada}

\begin{itemize}
\item \textsc{bosi}, Alfredo (org.). \emph{O conto brasileiro contemporâneo}. 16 ed.
São Paulo: Cultrix, {[}1975{]} 2015.

Conforme o organizador indica no
prefácio, ``Situação e formas do conto brasileiro contemporâneo'' (pp.
7--22), a antologia possibilita aos leitores conhecerem a variedade do
gênero, entre realismo, fantasia e experimentação, na produção
brasileira posterior ao modernismo, caminho pavimentado por Monteiro
Lobato. Notas biobibliográficas acerca de seus autores acompanham os
textos selecionados.

\item \textsc{bosi}, Alfredo. \emph{Reflexões sobre a arte}. 4. ed. São Paulo: Ática,
1991.

Obra que apresenta o sentido da arte como combinação de três
dimensões: a vertente de construção formal, a de representação social e
a de expressão subjetiva, às quais se soma a vertente de transitividade
com o leitor. A observação de como essas dimensões se articulam nas
obras de arte constitui o exercício crítico, conforme Bosi empreendeu em
relação à fortuna crítica de Machado de Assis, no livro \emph{Brás Cubas
em três versões}, publicado pela Companhia das Letras em 2006. Boa
síntese se encontra no artigo-conferência ``Machado de Assis na
encruzilhada dos caminhos da crítica'' (\emph{Machado de Assis em
Linha}, ano 2, n. 4, dez. 2009, disponível em:
\url{http://machadodeassis.net/download/numero04/num04artigo02.pdf}).

\item \textsc{ferreira}, Aurélio Buarque de Holanda \& \textsc{rónai}, Paulo (organizadores e
tradutores). \emph{Mar de histórias: antologia do conto mundial}. 4 ed.
Rio de Janeiro: Nova Fronteira, {[}1978{]} 1999, 10 vols.

Reunindo
narrativas antigas e mais recentes, muitas célebres, outras traduzidas
pela primeira vez para a língua portuguesa, constitui a mais completa
panorâmica do conto universal.

\item \textsc{gotlib}, Nádia Battella. \emph{Teoria do conto}. São Paulo: Ática, 1985.

Com base nas reflexões teóricas de Vladimir Propp, Edgar Allan Poe,
Anton Tchekhov e Julio Cortázar, e na análise de textos curtos porém
plenos de significado de grandes contistas, a professora Nádia,
livre-docente em Literatura Brasileira pela Universidade de São Paulo,
leva os leitores a conhecerem e compreenderem a especificidade do conto
como gênero literário.

\item \textsc{poe}, Edgar A. ``A filosofia da composição'' {[}\emph{The Philosophy of
Composition}, 1846{]}. In: \emph{Ficção completa, poesia \& ensaios}.
Organização, tradução e notas por Oscar Mendes, com a colaboração de
Milton Amado. Rio de Janeiro: Aguilar, 1981, pp. 911--20; \textsc{poe}, Edgar A.
\emph{A filosofia da composição}. Prefácio de Pedro Süssekind. Tradução
de Léa Viveiros de Castro. 2 ed. Rio de Janeiro: 7Letras, 2011.

Texto fundamental para o estudo do conto: analisando ``O corvo'', seu mais
famoso poema, Poe apresenta seu processo consciente de composição e
desenvolve uma teoria baseada no princípio da unidade de efeito.

\item \textsc{propp}, Vladimir. \emph{Morfologia do conto maravilhoso}. Tradução de
Jasna Paravich Sarhan.

Organização e prefácio de Boris Schnaiderman. 2.
ed. Rio de Janeiro: Forense Universitária, 2006. Propp se dedica à
descrição de contos populares russos, formados por esquemas narrativos
constantes, em busca de conhecer sua estrutura e de definir o conto
maravilhoso.

\item \textsc{ramos}, Graciliano. \emph{Contos e novelas}. Rio de Janeiro:
Livraria-Editora da Casa do Estudante do Brasil, 1957, 3 vols.: Norte e
Nordeste; Leste; Sul e Centro-Oeste; \emph{Seleção de contos
brasileiros}. Rio de Janeiro: Edições de Ouro, 1966. 3 vols.: Norte e
Nordeste; Leste; Sul e Centro-Oeste.

Antologia organizada pelo escritor
Graciliano Ramos nos anos 1940, com base em pesquisa realizada na
Academia Brasileira de Letras e na Biblioteca Nacional: segue um
critério geográfico, incluindo escritores antigos e modernos de todo o
país. Entre os contos escolhidos por Graciliano estão ``Tragédia dum
capão de pintos'', de Monteiro Lobato, ``Os brincos de Sara'', de
Alberto de Oliveira, e ``Útil inda brincando'', de Artur Azevedo, aqui
presentes.

\item \textsc{sanso}, Flávio. \emph{Viva Ludovico}. São Paulo: Quatro Cantos, 2019.

\emph{Viva Ludovico} conta a história de um açougueiro que, embora
acostumado a abater bois, de repente se afeiçoa a um deles e luta por
sua vida. O leitor é levado a refletir sobre a relação homem/animal,
sociedade e poder.

\item \textsc{schopenhauer}, Arthur. \emph{Sobre o fundamento da moral}. Tradução de
Maria Lúcia Mello Oliveira Cacciola. 2 ed. São Paulo: Martins Fontes,
2001. Coleção Clássicos.

Segundo ensina Schopenhauer, a compaixão
depende de ``o animal espectador identificar-se com o animal que
sofre'', constituindo ``a própria motivação moral''.
\end{itemize}
