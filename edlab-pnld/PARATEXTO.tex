\chapter{Uma antologia de contos do cotidiano: temas e gênero}
 
Os contos desta antologia foram dispostos em cinco
seções, conforme assuntos do cotidiano, do universo escolar e de
dimensões afetivas, políticas e culturais, que certamente vão despertar
o interesse dos leitores: ``Na Escola'', ``Na Realidade Social'', ``Nas
Relações Amorosas'', ``Na Língua'' e ``Entre Animais''.

Todos os textos aqui reunidos levam a pensar sobre os sentidos de iludir
o outro ou de iludir-se, fatos dos mais corriqueiros. Interessa
considerar que a palavra \emph{ilusão} se forma de \emph{in-},
``inclusão'', e \emph{ludere}, ``jogo, brincadeira''; e que significava
originalmente ironia e, depois, ``erro de percepção'', ``engano da
mente''. O aspecto lúdico, de brincadeira, da palavra \emph{ilusão} será
resgatado no prazer da leitura, afinal algumas ilusões representadas nos
contos provocam confusões engraçadas. Mas também situações trágicas
estão associadas a ilusões retratadas em outros textos, e então o gosto
da leitura vem da comoção sentida, do sentimento de compaixão pelos que
sofrem, o qual é fundamento da moral.

Trata-se de \emph{contos}, ou seja, narrativas breves que apresentam um conflito e uma
única ação voltada a causar efeito no leitor, geralmente com unidade de
espaço e de tempo, e número restrito de personagens (ao contrário, por exemplo, dos romances, que, de forma geral, costumam ser mais extensos, com múltiplos núcleos narrativos, envolvendo muitas personagens, ampliando arcos de tempo, em diversos espaços). Narrar pressupõe um
ponto de vista, o viés com que o narrador conta a história, o qual
evidentemente interfere na forma de abordar o assunto. Lendo os contos,
você poderá compreender situações diversas envolvendo personagens e
narradores diferentes, e identificar-se ou não com eles. Em contato com
as figuras e os estilos dos textos ficcionais, ao perceber diferenças
sociais, de comportamento, de expressão e de percepção de si e do outro,
você terá um ganho de consciência da realidade, de si mesmo e da
linguagem.

O interesse da primeira seção deste livro transparece no título: ``Na
Escola''. Histórias atraentes, centradas no ambiente escolar, mostram a
necessidade de que os alunos estejam bem-alimentados, calçados, e de que
haja respeito entre os colegas, professores e funcionários. A reflexão e
o debate acerca de temas como cola, inveja, preconceitos, a importância
de uma boa formação intelectual e ética, o gosto autêntico pelos estudos
e pela profissão serão propiciados por contos de grandes nomes da nossa
literatura: ``As calças do Raposo'', de Medeiros e Albuquerque; ``Conto
de escola'', de Machado de Assis; ``Um bom diretor'', Lima Barreto; e
``Pé no chão --- Vidinha ociosa'', de Monteiro Lobato.

Os contos da segunda seção, ``Na Realidade Social'', trazem ilusões e
confusões no plano das relações sociais, em especial referentes ao mundo
do trabalho e à realidade de desigualdades econômicas. A busca por um
bom emprego, o desejo de ajudar financeiramente a família, os
obstáculos, disfarces, bajulações, rasteiras e enganos da vida em
sociedade, que provocam risos e lágrimas, podem ser encontrados nos
contos ``Dona Expedita'', ``O fisco (Conto de Natal)'' e ``Os
pequeninos'', de Monteiro Lobato; ``Galeria póstuma'', de Machado de
Assis, e ``O homem que sabia javanês'', Lima Barreto.

É fácil imaginar que há ilusões, desilusões e confusões ``Nas Relações
Amorosas'', conforme o título da terceira seção deste livro. Então, os
leitores se surpreenderão com a construção dos enredos e os desfechos de
``Primas de Sapucaia!'' e ``A cartomante'', de Machado de Assis; ``Os
brincos de Sara'', de Alberto de Oliveira; e ``Útil inda brincando'', de
Artur Azevedo. Em especial nesses contos, vale observar como o ponto de
vista do narrador pode criar suspense e levar a refletir sobre relações
marcadas por paixão, amor, violência, traição.

A seção ``Na Língua'', com os contos de Artur Azevedo
``Plebiscito'' e ``As asneiras do Guedes'', e a última seção, ``Entre
Animais'', com ``Tragédia dum capão de pintos'', de Monteiro Lobato, e
``Ideias de canário'', de Machado de Assis, provocam o riso ou lágrimas,
mas sempre despertam a consciência crítica contra a ignorância: contra o
desconhecimento da língua disfarçado em sabedoria, contra a
insensibilidade para com os animais e os diferentes, contra as posturas
inflexíveis, os absolutismos.

No caminho dos contos entre ilusões e confusões, interessa a
palavra \emph{quiproquó}: vem da expressão latina \emph{quid pro quo},
``uma coisa pela outra''. Por exemplo, em ``Dona Expedita'', o
mal-entendido, o engano de tomar uma coisa por outra, causa o riso,
embora exista um fundo trágico, que é atualíssimo, a situação de
desemprego e desigualdade social. E contos como ``Os pequeninos'',
``Tragédia dum capão de pintos, ``Ideias de canário'', ``Primas de
Sapucaia!'', ao presentificarem relações entre animais, entre homens e
animais, relações de trabalho e de amor, apontam para a relatividade dos
pontos de vista e se abrem para estudos interdisciplinares, envolvendo
diversas áreas, como a literatura, a história, a filosofia, as
ciências.

Assim, tal síntese indica o valor artístico desta antologia e
seu potencial para contemplar a temática ``Sociedade, política e
cidadania'', de relevância ética para a formação dos alunos.

\chapter{Sobre os autores} %Contextualização dos autores

Você certamente já ouviu o nome de um dos mais importantes
escritores brasileiros: Machado de Assis. Aqui você conhecerá estes
contos criados por ele: ``Conto de escola'', ``Galeria póstuma'',
``Primas de Sapucaia!'', ``A cartomante'' e ``Ideias de canário''.

\subsection{Machado de Assis}

Joaquim Maria Machado de Assis (Rio de Janeiro, Rio de Janeiro,
1839--Rio de Janeiro, Rio de Janeiro, 1908) foi jornalista, romancista,
contista, cronista, poeta e teatrólogo. Fundador da cadeira n. 23 da
Academia Brasileira de Letras, escolheu José de Alencar, que morrera
cerca de vinte anos antes da fundação da \textsc{abl}, para seu patrono. Ocupou
por mais de dez anos a presidência da Academia, que passou a ser chamada
também de Casa de Machado de Assis.

Filho do pintor afrodescendente Francisco José de Assis e da açoriana
Maria Leopoldina Machado de Assis, perdeu a mãe muito cedo e foi criado
no morro do Livramento. Autodidata, aos quinze anos incompletos publicou
o soneto ``À Ilma. Sra. D.P.J.A.'', seu primeiro trabalho literário,
no~\emph{Periódico dos Pobres}. Aprendiz de tipógrafo, entrou em 1856
para a Imprensa Nacional, onde conheceu o escritor Manuel Antônio de
Almeida, que se tornou seu protetor. Em 1858, era revisor e colaborador
no~\emph{Correio Mercantil}~e, dois anos depois, a convite de Quintino
Bocaiuva, entrou para a redação do~\emph{Diário do Rio de Janeiro}. Na
revista~\emph{O Espelho}, estreou como crítico teatral, colaborou também
na~\emph{Semana Ilustrada}~e no~\emph{Jornal das Famílias}, sobretudo
com contos.

Estampada pela primeira vez em \emph{A Marmota}, de 19 de abril a 3 de
maio de 1861, periódico editado por Paula Brito, a tradução
de~\emph{Queda que as mulheres têm para os tolos}~(1861) foi o primeiro
livro publicado por Machado de Assis. Em 1862, foi censor teatral:
embora o cargo não fosse remunerado, garantia-lhe ingresso nos teatros.
Começou a colaborar em~\emph{O Futuro}, órgão dirigido por Faustino
Xavier de Novais, irmão de Carolina Augusta Xavier de Novais, sua futura
esposa, com quem se casou em novembro de 1869.

Em 1864, saiu o primeiro livro de poesia de Machado, \emph{Crisálidas};
e em 1872, o primeiro romance,~\emph{Ressurreição}. No ano seguinte, ele
foi nomeado primeiro oficial da Secretaria de Estado do Ministério da
Agricultura, Comércio e Obras Públicas. Era o início da carreira de
burocrata, sua fonte principal de sustento. Em 1874,~o romance~\emph{A
mão e a luva} foi estampado em \emph{O Globo}, em folhetins.
Intensificando sua colaboração em periódicos como~\emph{O
Cruzeiro},~\emph{A Estação},~\emph{Revista Brasileira}, Machado publicou
contos, romances, crônicas e poemas em folhetins e depois em livros. Sua
peça~\emph{Tu, só tu, puro amor} foi encenada no Imperial Teatro Dom
Pedro \textsc{ii} em 1880, em comemoração ao tricentenário de Camões. De 1881 a
1897, publicou na~\emph{Gazeta de Notícias}~suas melhores crônicas. Em
1880, o poeta Pedro Luís Pereira de Sousa o convidou para oficial de
gabinete no Ministério da Agricultura, Comércio e Obras Públicas. Em
1889, Machado foi promovido a diretor da Diretoria do Comércio no
Ministério em que servia.

De 15 de março a 15 de dezembro de 1880, as \emph{Memórias póstumas de
Brás Cubas} saíram em folhetins na~\emph{Revista Brasileira} e, no ano
seguinte, em livro. Tal obra inicia a chamada segunda fase da trajetória
literária de Machado de Assis. Escritor dedicado ao longo de 54 anos de
vida literária a praticamente todos os gêneros --- poesia, crítica,
teatro, tradução, crônica, conto e romance ---, ele exerceu em sua poesia
o romantismo em \emph{Crisálidas} (1864) e \emph{Falenas} (1870), o
indianismo em~\emph{Americanas}~(1875) e o parnasianismo
em~\emph{Ocidentais}~(1901). Dessa fase considerada romântica são também
as coletâneas \emph{Contos fluminenses}~(1870) e~\emph{Histórias da
meia-noite}~(1873) e os romances~\emph{Ressurreição}~(1872),~\emph{A mão
e a luva}~(1874),~\emph{Helena}~(1876) e~\emph{Iaiá Garcia}~(1878). A
partir daí, Machado de Assis criou obras-primas, que fogem a qualquer
denominação de escola literária e o tornaram o grande escritor
brasileiro.

Depois de \emph{Memórias póstumas de Brás Cubas} (1881), publicou os
romances \emph{Quincas Borba} (1890), \emph{Dom Casmurro} (1900),
\emph{Esaú e Jacó} (1904) e \emph{Memorial de Aires} (1908).
Diversamente do viés romântico de seus primeiros romances, essa segunda
fase sobressai por seu realismo pleno de ambiguidades e implacável, o
qual, ao analisar a psicologia das personagens em sociedade,
desestabiliza certezas do leitor e deixa ver a primazia da natureza
egoísta. Inclui volumes de contos como~\emph{Papéis avulsos}~(1882),
\emph{Histórias sem data} (1884), \emph{Várias histórias} (1896) e
\emph{Páginas recolhidas} (1899).

Em vida, a obra de Machado foi editada pela Livraria Garnier, desde
1869; em 1937, as~\emph{Obras completas}, em 31 volumes, saíram pela W.
M. Jackson, do Rio de Janeiro. Raimundo Magalhães Júnior organizou e
publicou, pela Civilização Brasileira, os seguintes volumes de Machado
de Assis:~\emph{Contos e crônicas}~(1958),~\emph{Contos
esparsos}~(1956),~\emph{Contos esquecidos}~(1956),~\emph{Contos
recolhidos}~(1956),~\emph{Contos avulsos}~(1956),~\emph{Contos sem
data}~(1956), Crônicas de Lélio (1958), Diálogos e reflexões de um
relojoeiro (1956).~

O conto que dá título a este livro foi escrito por um amigo de Machado
de Assis, chamado Medeiros e Albuquerque. Ele relata que
Machado tinha lido ``As calças do Raposo'' na \emph{Revista
Brasileira}. Dias depois, quando ia tomar o bonde, no Largo da Carioca,
viu um homem que lhe pareceu conhecido. Fez um grande esforço de memória
e conseguiu identificar quem era: ``--- Ah! É o Raposo, do Medeiros!''.
Ou seja, Machado visualizava com grande intensidade as figuras
literárias que encontrava nos livros. Imagine como deviam ser nítidas as
figuras que ele criava.

\subsection{Medeiros e Albuquerque}

José Joaquim de Campos da Costa de Medeiros e Albuquerque (Recife, Pernambuco, 1867--Rio de Janeiro, Rio de Janeiro, 1934) foi
romancista, contista, poeta, teatrólogo, ensaísta, memorialista,
jornalista, professor, político e orador. Usou diversos pseudônimos:
Armando Quevedo, Atásius Noll, J. dos Santos, Max, Rifiúfio Singapura.
Fundou a Cadeira número 22 da Academia Brasileira de Letras, cujo
patrono é José Bonifácio, o Moço.

Cursou o Colégio Pedro \textsc{ii} e, de 1880 até 1884, a Escola Acadêmica, em
Lisboa. De volta ao Rio de Janeiro, fez um curso de História Natural com
Emílio Goeldi e foi aluno particular do crítico Sílvio Romero. Trabalhou
como professor primário adjunto, entrando em contato com escritores como
Paula Ney e Pardal Mallet. Em 1888, colaborou no jornal
\emph{Novidades}. Estreou em 1889 com os livros de poesia \emph{Pecados}
e \emph{Canções da decadência}.

Republicano, foi nomeado secretário do Ministério do Interior e
vice-diretor do Ginásio Nacional. Lecionou na Escola de Belas Artes
(desde 1890) e em escolas de segundo grau, de 1890 a 1897, além de ter
sido presidente do Conservatório Dramático (1890--1892). É o autor da
letra do Hino da República, cujo refrão é: ``Liberdade! Liberdade!/ Abre
as asas sobre nós/ Das lutas na tempestade/ Dá que ouçamos tua voz''.
Durante o período florianista, dirigiu o jornal \emph{O Fígaro}. Em
1894, foi eleito deputado federal por Pernambuco, conseguindo a votação
para a lei dos direitos autorais. Nomeado diretor geral da Instrução
Pública do Distrito Federal em 1897, por estar na oposição a Prudente de
Morais, teve de pedir asilo à Embaixada do Chile. Demitido, recorreu aos
tribunais e obteve a reintegração. De 1912 a 1916, viveu em Paris. De
volta ao Brasil, de 1899 a 1917 ocupou a Secretaria Geral da \textsc{abl}. Foi
autor da primeira reforma ortográfica, promovida em 1902. De 1930 a
1934, colaborou na \emph{Gazeta} de São Paulo e em jornais do Rio de
Janeiro e na Comissão do Dicionário e na \emph{Revista da Academia}.

Principais obras: poesia: \emph{Pecados} (1889); \emph{Canções
da decadência} (1889); \emph{Poesias}, 1893--1901 (1904); \emph{Fim}
(1922); \emph{Poemas sem versos} (1924); contos: \emph{Um homem prático}
(1898); \emph{Mãe Tapuia} (1900); \emph{Contos escolhidos} (1907);
\emph{O assassinato do general} (1926); \emph{Se eu fosse Sherlock
Holmes} (1932); romances: \emph{Marta} (1920); \emph{Mistério} (1921);
\emph{Laura} (1933); teatro: \emph{O escândalo}, drama (1910) e
\emph{Teatro meu\ldots{} e dos outros} (1923); ensaios e conferências:
\emph{O silêncio é de ouro} (1912); \emph{Pontos de vista} (1913);
\emph{O hipnotismo} (1921); \emph{Graves e fúteis} (1922);
\emph{Literatura alheia} (1914); \emph{Páginas de crítica} (1920);
\emph{Homens e coisas da Academia} (1934); memórias: \emph{Minha vida:
da infância à mocidade, 1867--1893} (1933); \emph{Minha vida: da mocidade
à velhice, 1893--1934} (1934); \emph{Quando eu era vivo\ldots{}}
Memórias, 1867 a 1934 (1942); polêmicas e política: \emph{Polêmicas}.
Coligidas e anotadas por Paulo de Medeiros e Albuquerque (1941);
\emph{Parlamentarismo e presidencialismo} (1932).

\subsection{Artur Azevedo}

Outro amigo de Machado de Assis presente neste livro é Artur Azevedo: o
autor de ``Plebiscito'', de ``Útil inda brincando'' e de ``As asneiras
do Guedes'' conviveu com Machado durante 34 anos de trabalho na mesma
repartição pública. Comoveu-se muito com a morte do amigo, de quem
apreciava o espírito alegre e vivaz, que se abatera profundamente quando
da perda de Carolina, em 1904. Artur Azevedo considerava a obra
machadiana ``o melhor do nosso patrimônio literário'', um tesouro a ser
guardado e transmitido às gerações futuras.

Artur Nabantino Gonçalves de Azevedo (São Luís,
Maranhão, 1855--Rio de Janeiro, Rio de Janeiro, 1908) foi jornalista,
contista, teatrólogo e poeta. Figurou, ao lado do irmão, o romancista
Aluísio Azevedo, no grupo fundador da \textsc{abl} e criou a Cadeira número 29,
que tem como patrono o também dramaturgo Martins Pena. Fundou
publicações literárias, como \emph{A Gazetinha}, \emph{Vida Moderna} e
\emph{O Álbum}. Colaborou em \emph{A Estação}, ao lado de Machado de
Assis, e no jornal \emph{Novidades}, em que seus companheiros eram
Alcindo Guanabara, Moreira Sampaio, Olavo Bilac e Coelho Neto. Escreveu
principalmente sobre teatro em \emph{O País}, no \emph{Diário de
Notícias} e em \emph{A Notícia}. Utilizou diversos pseudônimos: Elói o
Herói, Gavroche, Petrônio, Cosimo, Juvenal, Dorante, Frivolino, Batista
o trocista, e outros. Principais obras de teatro: \emph{Amor por
anexins} (1872); \emph{A pele do lobo} (1877); \emph{A}
\emph{almanjarra} (1888); \emph{A Capital Federal} (1897);
\emph{Confidências} (1898); \emph{O jagunço} (1898). Poesias:
\emph{Sonetos} (1876); Rimas (1909). Contos: \emph{Contos possíveis}
(1889); \emph{Contos fora de moda} (1894); \emph{Contos efêmeros}
(1897); \emph{Contos cariocas} (1928).

\subsection{Alberto de Oliveira}

Também próximo de Machado de Assis, o quarto escritor aqui presente,
Alberto de Oliveira, contou com uma introdução escrita por ele no livro
de poesia \emph{Meridionais}, publicado pela Tipografia da Gazeta de
Notícias, do Rio de Janeiro, em 1884. O crítico Machado avalia Alberto
como um dos melhores de sua geração, mas não deixa de apontar defeitos
na poesia do amigo, atribuindo-os em parte ao excesso de
aperfeiçoamento.

Antônio Mariano Alberto de Oliveira (Saquarema, Rio de Janeiro,
1857--Niterói, Rio de Janeiro, 1937) foi poeta, professor e
farmacêutico. Atuou como secretário estadual de educação, foi membro
honorário da Academia de Ciências de Lisboa e fundador da Academia
Brasileira de Letras. Formado em Farmácia em 1884, cursou até o terceiro
ano a Faculdade de Medicina. Havendo estreado com \emph{Canções
românticas}, já nas \emph{Meridionais} (1884) e nas quatro séries de
\emph{Poesias} (1900, 1905, 1913 e 1928) filia-se ao parnasianismo, bem
como Raimundo Correia e Olavo Bilac. Eleito, em 1924, ``príncipe dos
poetas brasileiros''. Fundador da Cadeira 18 da Academia Brasileira de
Letras. Publicações: \emph{Canções românticas} (1878);
\emph{Meridionais} (1884); \emph{Relatório do Diretor da Instrução do
Estado do Rio de Janeiro} (1893, 1895); \emph{Versos e rimas} (1895);
\emph{Poesias} (1900, 1905, 1913 e 1928); \emph{O culto da forma na
poesia brasileira} (1916).

Como se viu, os quatro escritores, Machado de Assis, Medeiros e
Albuquerque, Artur Azevedo e Alberto de Oliveira, fazem parte do grupo
de fundadores da Academia Brasileira de Letras. A fundação da \textsc{abl}, a 28
de janeiro de 1897, contou também com Araripe Júnior, Coelho
Neto, Graça Aranha, Inglês de Sousa, Joaquim Nabuco, José do Patrocínio,
José Veríssimo, Lúcio de Mendonça, Luís Murat, Olavo Bilac, Rodrigo
Otávio, Rui Barbosa, Visconde de Taunay, entre outros.

Os demais contistas desta antologia, Monteiro Lobato e Lima Barreto,
nascidos respectivamente em 1882 e 1881, justamente o ano de publicação
das \emph{Memórias póstumas de Brás Cubas}, são, portanto, de outra
geração. Se a historiografia enquadra os primeiros na escola realista,
tendo em Machado de Assis seu principal representante, costuma
considerar estes, Lobato e Lima, como pré-modernistas. No entanto, tal
denominação, mesmo reconhecendo neles seu teor crítico e revolucionário,
peca por certo anacronismo, ao projetar valor na obra desses autores por
um sentido antecipador dos modernistas de 1922.

\subsection{Monteiro Lobato}

José Bento Renato Monteiro Lobato (Taubaté, São Paulo, 1882--São Paulo, São Paulo, 1948) foi pioneiro no Brasil como criador da
literatura infantojuvenil e editor, além de ter sido contista,
jornalista, tradutor, pintor e fotógrafo. Aos onze anos, mudou seu nome
para José Bento, por causa das iniciais gravadas no castão da bengala do
pai, J.B.M.L. Apesar de sua inclinação para as artes plásticas, cursou a
Faculdade de Direito do Largo São Francisco, em São Paulo, por imposição
do avô, o Visconde de Tremembé.

Formado em 1904, voltou a Taubaté, onde foi nomeado promotor público
interino e transferido, em 1907, para Areias, São Paulo. Enviou artigos
para \emph{A Tribuna}, de Santos, traduções para o jornal \emph{O Estado
de S. Paulo} e caricaturas para a revista \emph{Fon-Fon!}, do Rio de
Janeiro. Em 1911, herdou, com as duas irmãs, a fazenda do avô. Publicou,
em 1914, os artigos ``Velha praga'' e ``Urupês'' em \emph{O Estado de S.
Paulo}, criando o personagem Jeca Tatu. Em 1917, vendeu a fazenda e se
mudou para São Paulo. Escreveu, em \emph{O Estado de S. Paulo}, o artigo
``A propósito da Exposição de Malfatti'', conhecido como ``Paranoia ou
mistificação?'', que abriu polêmica com os modernistas.

Em 1918, estreou com o livro de contos \emph{Urupês}, que esgotou 30 mil
exemplares entre 1918 e 1925, e comprou a \emph{Revista do Brasil},
lançando as bases da indústria editorial no país. Criando uma rede de
distribuição, com vendedores autônomos e consignatários, revolucionou o
mercado livreiro. Em 1920, fundou a editora Monteiro Lobato \& Cia. E
lançou \emph{A menina do narizinho arrebitado}, primeira da série de
histórias com que Lobato criou a literatura brasileira dedicada às
crianças, formando gerações de leitores. Em 1924, com capital ampliado e
nova denominação, Companhia Gráfico-Editora Monteiro Lobato, sua editora
montou o maior parque gráfico da América Latina. Porém, no ano seguinte,
dificuldades financeiras o levaram a vender a \emph{Revista do Brasil} e
liquidar a editora. Mudou-se para o Rio de Janeiro e fundou a Companhia
Editora Nacional.

Adido comercial em Nova York de 1927 até 1930, voltou ao Brasil com
ideias para a exploração de ferro e petróleo. Fundou empresas de
prospecção, mas, contrariando interesses multinacionais e fazendo
oposição, em artigos e entrevistas, ao governo Vargas, foi preso por
seis meses em 1941. Recebeu indulto depois de cumprir metade da pena,
mas o governo mandou apreender e queimar seus livros infantis. Em 1944,
recusou indicação para a Academia Brasileira de Letras. Em 1946,
tornou-se sócio da editora Brasiliense. Embarcou para a Argentina e
fundou em Buenos Aires a Editorial Acteon, retornando no ano seguinte a
São Paulo.

Principais obras: livros para crianças: \emph{O Saci} (1921);
\emph{Fábulas} (1922); \emph{Reinações de Narizinho} (1931);
\emph{Viagem ao céu} (1932); \emph{Caçadas de Pedrinho} (1933);
\emph{História do Mundo para as} \emph{Crianças} (1933); \emph{Emília no
País da Gramática} (1934); \emph{Aritmética da Emília} (1935);
\emph{Memórias da Emília} (1936); \emph{O Poço do Visconde} (1937);
\emph{O Picapau Amarelo} (1939); \emph{A Reforma da Natureza} (1941);
\emph{A Chave do Tamanho} (1942); \emph{Os doze trabalhos de Hércules},
dois volumes (1944). Livros para adultos: \emph{Urupês} (1918);
\emph{Cidades} \emph{mortas} (1919); \emph{Ideias de Jeca Tatu} (1919);
\emph{Negrinha} (1920); \emph{Mundo da lua} (1923); \emph{O Presidente
Negro/O choque das raças} (1926); \emph{Ferro} (1931); \emph{América}
(1932); \emph{O escândalo do petróleo} (1936); \emph{A barca de Gleyre}
(1944).

Monteiro Lobato exaltou a ``inteligência criadora'' de Machado de Assis,
que, tendo ascendido socialmente, foi capaz de observar os mecanismos da
sociedade e atingiu a ``intuição perfeita de tudo''. Ele declarou, com
bom humor, que, diante de Machado, ``somos todos uns bobinhos''. Já Lima
Barreto, embora reconhecesse os méritos de grande escritor do criador de
Brás Cubas, tinha ressalvas a sua secura de alma e a sua falta de
simpatia.

\subsection{Lima Barreto}

Afonso Henriques de Lima Barreto (Rio de Janeiro, Rio de
Janeiro, 1881--Rio de Janeiro, Rio de Janeiro, 1922) foi jornalista e
escritor. Filho de um tipógrafo e de uma professora primária, ambos
descendentes de escravos, ficou órfão de mãe aos sete anos. Proclamada a
República, seu pai foi demitido da Imprensa Nacional, tendo lá entrado
pela mão do Visconde de Ouro Preto. Lima Barreto estudou no Colégio
Pedro \textsc{ii} e ingressou na Escola Politécnica, mas abandonou o curso de
Engenharia, quando seu pai enlouqueceu e foi internado. Para sustentar a
família, passou a trabalhar como amanuense na Secretaria da Guerra em
1903 e colaborava em diversos jornais do Rio de Janeiro. Em 1905,
começou a escrever no \emph{Correio da Manhã}, jornal de grande
prestígio. Editou a \emph{Revista Floreal}, nos fins de 1907, e foi
colaborador das revistas \emph{Fon-Fon}, \emph{A.B.C.} e \emph{Careta}.

Iniciou sua carreira de romancista em 1909, com \emph{Recordações do
escrivão Isaías Caminha}, tido pela crítica dos contemporâneos tão só
como um romance \emph{à} \emph{clef}, isto é, em que o autor trata de pessoas reais por meio de personagens fictícios". Dois anos depois, \emph{Triste fim
de Policarpo Quaresma} sai em folhetim no \emph{Jornal do Commercio} e,
publicada em livro em 1915, tem recepção elogiosa. Em 1919, Lima publica
\emph{Vida e morte de M. J. Gonzaga de Sá}, pela editora Revista do
Brasil, de Monteiro Lobato. Vivendo crises de depressão e entregando-se
à bebida, internou-se no Hospício Nacional em 1914 e em 1919, daí ter
escrito \emph{Cemitério dos vivos}. A partir de 1918, passou a militar
na imprensa maximalista.

Muitos de seus escritos foram redescobertos e publicados em livro após
sua morte, por Francisco de Assis Barbosa e outros pesquisadores.
Principais obras: \emph{Recordações} \emph{do} \emph{escrivão}
\emph{Isaías Caminha} (1909); \emph{As aventuras do Dr. Bogoloff}
(1912); \emph{Triste fim de Policarpo Quaresma} (1915); \emph{Numa e a
ninfa} (1915); \emph{Vida e Morte de M. J. Gonzaga de Sá} (1919);
\emph{Histórias e sonhos} (1920); \emph{Os Bruzundangas} (1923);
\emph{Clara dos Anjos} (póstumo, 1948); \emph{Diário íntimo} (1953);
\emph{Feiras e Mafuás} (1953); \emph{Cemitério dos vivos} (póstumo e
inacabado, 1956).

\chapter{Sobre as obras} %Contextualização das obras

Quanto ao contexto histórico do país na época de Machado de Assis,
Medeiros e Albuquerque, Artur Azevedo e Alberto de Oliveira, destacam-se
a Abolição da escravatura e a proclamação da República como fatos
centrais. Desde a extinção do tráfico de escravos, em 1850, aumentou a
decadência da economia açucareira, deslocou-se o eixo de prestígio para
o Sul, e se intensificaram as ideias liberais, abolicionistas e
republicanas. Porém, como se sabe, a Abolição não resolveu o problema da
espoliação dos ex-escravos, e tal origem colonial do Brasil trouxe como
consequências a violência, a desigualdade social, a educação precária
que marcam o país.

Em termos de sociabilidade literária, a ideia de criar a Academia
Brasileira de Letras veio do grupo de intelectuais que se reunia na
redação da~\emph{Revista Brasileira}, sobretudo de Lúcio de Mendonça.
Machado de Assis compareceu às reuniões preparatórias e, quando se
fundou a Academia, a 28 de janeiro de 1897, foi eleito presidente,
tendo-se dedicado a ela até o fim da vida.

Destaque-se que os escritores aqui reunidos colaboraram bastante na
imprensa, e diversos de seus contos saíram primeiro em periódicos, mais
acessíveis ao público, e depois no formato livro, às vezes sofrendo
modificações. Machado de Assis, por exemplo, publicou nos inícios da
edição e do comércio livreiro no Brasil, com os pioneiros da atividade
no país: Francisco de Paula Brito, os irmãos Garnier, os irmãos Laemmert
e Henri Lombaerts.

Seus contos ``Galeria póstuma'' e ``Primas de Sapucaia!'' foram
estampados na \emph{Gazeta de Notícias}, respectivamente a 2 de agosto e
24 de outubro de 1883, e no ano seguinte incluídos no volume
\emph{Histórias sem data}, publicado por B. L. Garnier --
Livreiro-Editor.

No mesmo periódico saíram ``Conto de escola'' e ``A cartomante'',
respectivamente a 8 de setembro e 28 de novembro de 1884, depois
publicados em \emph{Várias histórias} (1896), por Laemmert \& C.
Editores.

E ``Ideias de canário'' também foi lido na \emph{Gazeta de Notícias}, em
15 de janeiro de 1895, e depois em \emph{Páginas recolhidas} (1899), de
H. Garnier, Livreiro-Editor.

Esses contos de Machado constam do volume 2 de sua \emph{Obra completa
em três volumes} (Org. Afrânio Coutinho. Rio de Janeiro: Nova Aguilar,
2004).

Já o conto ``As calças do Raposo'', de Medeiros e Albuquerque, foi
estampado na \emph{Revista Brasileira}, do Rio de Janeiro, na edição de
janeiro-março de 1899 (ano V, tomo \textsc{xvii}, pp. 319--34). E depois incluído
no volume \emph{Mãe Tapuia: contos}, publicado no Rio de Janeiro por H.
Garnier, Livreiro Editor, em 1900.

Os engraçados causos narrados por Artur Azevedo ``Útil inda brincando''
e ``Plebiscito'' constam de \emph{Contos fora da moda}, publicados em
1893 pela Garnier, com segunda edição em 1901. ``Útil inda brincando''
saiu em \emph{O Álbum}, do Rio de Janeiro, em abril de 1893 (pp.
108--111), e em \emph{O Fluminense}, de Niterói, a 11 e 12 de janeiro de
1899. Curiosamente, a expressão ``Útil inda brincando'', bastante comum
na época, como se vê nos jornais, era legenda de uma escultura no
chafariz dos Jacarés, a Fonte do Menino, no Passeio Público do Rio de
Janeiro. O menino da fonte era lúdico e útil, por fornecer água a todos.

Antes de sair em livro, ``Plebiscito'' foi publicado em \emph{O País},
do Rio de Janeiro, a 4 de abril de 1890, em \emph{O Republicano} de
Sergipe, a 3 de maio de 1890, e em \emph{O Republicano} de Cuiabá, a 9
de fevereiro de 1896.

Já ``As asneiras do Guedes'' saiu em \emph{A República}, do Rio de
Janeiro, a 13 de julho de 1894, e depois, em 1928, no volume póstumo
\emph{Contos cariocas} (Rio de Janeiro: Leite Ribeiro).

De Alberto de Oliveira, o conto ``Os brincos de Sara'' foi publicado na
\emph{Gazeta de Notícias}, do Rio de Janeiro, a 20 de junho de 1892. Foi
escolhido por Graciliano Ramos para figurar na coletânea \emph{Contos e
novelas} (Rio de Janeiro: Livraria-Editora da Casa do Estudante do
Brasil, 1957, 3 vols.: Norte e Nordeste; Leste; Sul e Centro-Oeste;
\emph{Seleção de contos brasileiros}. Rio de Janeiro: Edições de Ouro,
1966, 3 vols.; vol. \textsc{ii}, Leste).

De Lima Barreto, ``Um bom diretor'' saiu na revista \emph{Careta}, do
Rio de Janeiro, em 3 de abril de 1915, e depois no volume
\emph{Histórias e sonhos. Contos} (São Paulo: Brasiliense, 1961). E ``O
homem que sabia javanês'' foi estampado na \emph{Gazeta da Tarde}, do
Rio de Janeiro, em 28 de abril de 1911, e em 1915, como Apêndice da
primeira edição de \emph{Triste fim de} \emph{Policarpo Quaresma} (Rio
de Janeiro: Tipografia ``Revista dos Tribunais'').

Por fim, de Monteiro Lobato, o texto ``Pé no chão'', de ``Vidinha
ociosa'' (1908), consta de \emph{Cidades mortas}, publicado em 1919,
Edição da Revista do Brasil, de São Paulo. O conto ``Tragédia dum capão
de pintos'' saiu na edição de julho de 1923 da \emph{Revista do Brasil}
e foi recolhido no volume \emph{O macaco que se fez homem}, publicado no
mesmo ano pela editora Monteiro Lobato, de São Paulo.

``Dona Expedita'', de 1939, foi incluído na segunda edição de
\emph{Negrinha}, volume 3 das \emph{Obras completas} de Lobato,
publicadas em 1946 pela Brasiliense, de São Paulo. À primeira edição
desse livro, de 1920, composta dos contos ``Negrinha'', ``As fitas da
vida'', ``O drama da geada'', ``Bugio moqueado'', ``O jardineiro
Timóteo'' e ``O colocador de pronomes'', Lobato acrescentou ``Dona
Expedita'', ``O fisco (Conto de Natal)'' e ``Os pequeninos'', aqui
presentes, bem como ``Os negros'', ``Barba Azul'', ``Uma história de mil
anos'', ``A facada imortal, ``A policetemia de Dona Lindoca'', ``Duas
cavalgaduras'', ``O bom marido'', ``Marabá'', ``Fatia de vida'', ``A
morte do Camicego'', ``Quero ajudar o Brasil'', ``Sorte grande'' e
``Herdeiro de si mesmo''.